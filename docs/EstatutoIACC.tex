\documentclass[]{book}
\usepackage{lmodern}
\usepackage{amssymb,amsmath}
\usepackage{ifxetex,ifluatex}
\usepackage{fixltx2e} % provides \textsubscript
\ifnum 0\ifxetex 1\fi\ifluatex 1\fi=0 % if pdftex
  \usepackage[T1]{fontenc}
  \usepackage[utf8]{inputenc}
\else % if luatex or xelatex
  \ifxetex
    \usepackage{mathspec}
  \else
    \usepackage{fontspec}
  \fi
  \defaultfontfeatures{Ligatures=TeX,Scale=MatchLowercase}
\fi
% use upquote if available, for straight quotes in verbatim environments
\IfFileExists{upquote.sty}{\usepackage{upquote}}{}
% use microtype if available
\IfFileExists{microtype.sty}{%
\usepackage{microtype}
\UseMicrotypeSet[protrusion]{basicmath} % disable protrusion for tt fonts
}{}
\usepackage[margin=1in]{geometry}
\usepackage{hyperref}
\hypersetup{unicode=true,
            pdfborder={0 0 0},
            breaklinks=true}
\urlstyle{same}  % don't use monospace font for urls
\usepackage{natbib}
\bibliographystyle{apalike}
\usepackage{longtable,booktabs}
\usepackage{graphicx,grffile}
\makeatletter
\def\maxwidth{\ifdim\Gin@nat@width>\linewidth\linewidth\else\Gin@nat@width\fi}
\def\maxheight{\ifdim\Gin@nat@height>\textheight\textheight\else\Gin@nat@height\fi}
\makeatother
% Scale images if necessary, so that they will not overflow the page
% margins by default, and it is still possible to overwrite the defaults
% using explicit options in \includegraphics[width, height, ...]{}
\setkeys{Gin}{width=\maxwidth,height=\maxheight,keepaspectratio}
\IfFileExists{parskip.sty}{%
\usepackage{parskip}
}{% else
\setlength{\parindent}{0pt}
\setlength{\parskip}{6pt plus 2pt minus 1pt}
}
\setlength{\emergencystretch}{3em}  % prevent overfull lines
\providecommand{\tightlist}{%
  \setlength{\itemsep}{0pt}\setlength{\parskip}{0pt}}
\setcounter{secnumdepth}{5}
% Redefines (sub)paragraphs to behave more like sections
\ifx\paragraph\undefined\else
\let\oldparagraph\paragraph
\renewcommand{\paragraph}[1]{\oldparagraph{#1}\mbox{}}
\fi
\ifx\subparagraph\undefined\else
\let\oldsubparagraph\subparagraph
\renewcommand{\subparagraph}[1]{\oldsubparagraph{#1}\mbox{}}
\fi

%%% Use protect on footnotes to avoid problems with footnotes in titles
\let\rmarkdownfootnote\footnote%
\def\footnote{\protect\rmarkdownfootnote}

%%% Change title format to be more compact
\usepackage{titling}

% Create subtitle command for use in maketitle
\newcommand{\subtitle}[1]{
  \posttitle{
    \begin{center}\large#1\end{center}
    }
}

\setlength{\droptitle}{-2em}

  \title{}
    \pretitle{\vspace{\droptitle}}
  \posttitle{}
    \author{}
    \preauthor{}\postauthor{}
    \date{}
    \predate{}\postdate{}
  
\usepackage{booktabs}

\begin{document}

{
\setcounter{tocdepth}{1}
\tableofcontents
}
\chapter*{INSTITUTO ATLÉTICO CENTRAL
CÓRDOBA}\label{instituto-atletico-central-cordoba}
\addcontentsline{toc}{chapter}{INSTITUTO ATLÉTICO CENTRAL CÓRDOBA}

\textbf{ASOCIACIÓN CIVIL}

FUNDADO EL 8 DE AGOSTO DE 1918

ESTATUTOS SOCIALES

22/08/15

\chapter{DEL NOMBRE, DOMICILIO, FINES, DISTINTIVOS Y PATRIMONIO
SOCIAL}\label{del-nombre-domicilio-fines-distintivos-y-patrimonio-social}

\begin{itemize}
\tightlist
\item
  \textbf{Art. 1º} El Instituto Atlético Central Córdoba, fundado el 08
  de agosto de 1918, con sede legal en la Ciudad de Córdoba, Provincia
  del mismo nombre, República Argentina, fija su domicilio legal y sede
  administrativa en la calle Jujuy 2602, entre calles Lope de Vega y
  Calderón de la Barca, donde tendrá que ser ubicada la Secretaría y
  deben realizarse las Asambleas y reuniones de la Comisión Directiva.
  Es una Asociación Civil, con Personería Jurídica, que tiene por
  objeto:

  \begin{itemize}
  \item
    \begin{enumerate}
    \def\labelenumi{\alph{enumi})}
    \tightlist
    \item
      Propulsar el desarrollo de la cultura física, moral e intelectual
      de sus asociados, para lo cual habilitará las instalaciones
      deportivas y sociales que puedan proveerse de acuerdo con los
      medios y recursos que disponga.
    \end{enumerate}
  \item
    \begin{enumerate}
    \def\labelenumi{\alph{enumi})}
    \setcounter{enumi}{1}
    \tightlist
    \item
      Organizar competencias y torneos y participar en todos los actos
      conexos a sus fines, que organicen entidades civiles.
    \end{enumerate}
  \item
    \begin{enumerate}
    \def\labelenumi{\alph{enumi})}
    \setcounter{enumi}{2}
    \tightlist
    \item
      Promover el espíritu de unión y solidaridad entre sus adherentes.
    \end{enumerate}
  \item
    \begin{enumerate}
    \def\labelenumi{\alph{enumi})}
    \setcounter{enumi}{3}
    \tightlist
    \item
      Sostener relaciones con Instituciones Nacionales y Extranjeras,
      afines a sus propósitos y finalidades.
    \end{enumerate}
  \end{itemize}
\end{itemize}

\begin{itemize}
\tightlist
\item
  \textbf{Art. 2º} Los distintivos del Club serán:

  \begin{itemize}
  \item
    \begin{enumerate}
    \def\labelenumi{\alph{enumi})}
    \tightlist
    \item
      Bandera rojiblanca, en ocho franjas alternadas.
    \end{enumerate}
  \item
    \begin{enumerate}
    \def\labelenumi{\alph{enumi})}
    \setcounter{enumi}{1}
    \tightlist
    \item
      Casaca uniforme rojiblanca en listas verticales.
    \end{enumerate}
  \item
    \begin{enumerate}
    \def\labelenumi{\alph{enumi})}
    \setcounter{enumi}{2}
    \tightlist
    \item
      Escudo: Cinco (5) listas blancas verticales alternadas con cuatro
      (4) listas rojas y una lista transversal de izquierda a derecha y
      de abajo hacia arriba color blanco con las siglas I.A.C.C. en
      letras rojas.
    \end{enumerate}
  \end{itemize}
\end{itemize}

\begin{itemize}
\tightlist
\item
  \textbf{Art. 3º} La institución permanecerá ajena a todo dogma
  político, social y religioso y extraña a cualquier orientación racial
  o filosófica, para lo cual, en todas sus dependencias queda
  terminantemente prohibido promover y/o participar en cuestiones de
  carácter político, religioso o racial , como así mismo, organizar o
  realizar juegos de azar o apuestas, sin estar debidamente autorizadas.
\end{itemize}

\begin{itemize}
\tightlist
\item
  \textbf{Art. 4º} El patrimonio del club estará constituido por los
  bienes inmuebles, muebles, útiles y herramientas, créditos, rentas,
  trofeos, bienes intangibles tales como derecho al nombre, insignias y
  escudo, todos los fondos que ingresen a sus arcas cualquiera fuere su
  origen.
\end{itemize}

\begin{itemize}
\tightlist
\item
  \textbf{Art. 5º} La propiedad de los bienes que adquiera el Club,
  corresponderá a la institución como persona jurídica. Para ello, la
  Institución podrá realizar todos los actos jurídicos que, a juicio de
  sus órganos de representación, fueren necesarios o convenientes, entre
  otros:

  \begin{itemize}
  \item
    \begin{enumerate}
    \def\labelenumi{\alph{enumi})}
    \tightlist
    \item
      Adquirir, enajenar, prendar, permutar y locar toda clase de
      muebles o semovientes y arrendar toda clase de inmuebles en
      cualquier lugar de la República, como así también, adquirir,
      enajenar, prendar, hipotecar o permutar toda clase de inmuebles en
      cualquier parte del país, aceptar, constituir, transferir y
      extinguir hipotecas y todo derecho real, estas últimas acciones
      con autorización expresa de la Asamblea Societaria.
    \end{enumerate}
  \item
    \begin{enumerate}
    \def\labelenumi{\alph{enumi})}
    \setcounter{enumi}{1}
    \tightlist
    \item
      Realizar toda clase de operaciones con bancos o instituciones de
      crédito, públicas o privadas.
    \end{enumerate}
  \item
    \begin{enumerate}
    \def\labelenumi{\alph{enumi})}
    \setcounter{enumi}{2}
    \tightlist
    \item
      Aceptar donaciones, legados u otras liberalidades.
    \end{enumerate}
  \item
    \begin{enumerate}
    \def\labelenumi{\alph{enumi})}
    \setcounter{enumi}{3}
    \tightlist
    \item
      Celebrar contratos de locación de cosas, de obra, de trabajo, etc.
      Y realizar todos los demás actos y contratos relacionados con su
      objeto. La enumeración que antecede es meramente enunciativa y no
      limitativa, pudiendo en consecuencia hacer toda clase y tipo de
      operaciones que resuelvan sus autoridades legales, dentro de sus
      facultades, y se relacionen directa o indirectamente, con sus
      fines sociales.
    \end{enumerate}
  \end{itemize}
\end{itemize}

\begin{itemize}
\tightlist
\item
  \textbf{Art. 6º} La disolución de la Institución deberá resolverse en
  Asamblea General Extraordinaria, citada especialmente a ese efecto,
  con el voto favorable de los dos tercios (2/3) de los socios
  presentes, constituida el quórum mínimo del diez por ciento (10\%) de
  los asociados con derecho a voto; pero no podrá disponerse mientras
  exista en su seno un número de cincuenta asociados que, a la par de
  acreditar una permanencia consecutiva en el Club no inferior a cuatro
  años, resuelva encauzar y dirigir sus destinos dentro de los
  propósitos enunciados en este Estatuto.
\end{itemize}

\begin{itemize}
\tightlist
\item
  \textbf{Art. 7º} La Institución no tiene plazo determinado de
  duración. Para el caso de disolución, los asociados renuncian a todo
  derecho sobre los bienes sociales, los que pasarán de inmediato, una
  vez satisfecha la liquidación de sus deudas si las hubiere, a quien la
  Asamblea expresamente determinare, debiendo ser en todos los casos a
  favor de un ente de bien común, con personería jurídica.
\end{itemize}

\chapter{DE LOS ASOCIADOS}\label{de-los-asociados}

\begin{itemize}
\tightlist
\item
  \textbf{Art. 8º} El número de asociados de la Institución es ilimitado
  y podrá serlo cualquier persona de existencia visible que llene los
  requisitos exigidos a tal fin por el presente Estatuto.
\end{itemize}

\begin{itemize}
\tightlist
\item
  \textbf{Art. 9º} Quienes deseen incorporarse a la Institución deberán
  solicitarlo por escrito, acompañando referencia autenticada del
  documento de identidad y fotografías del aspirante y abonando la cuota
  de ingreso que se fije, y llenando los demás requisitos que se
  establezcan al efecto. Se establecen las siguientes categorías de
  socios (para ambos sexos):

  \begin{itemize}
  \item
    \begin{enumerate}
    \def\labelenumi{\alph{enumi})}
    \tightlist
    \item
      Honorarios.
    \end{enumerate}
  \item
    \begin{enumerate}
    \def\labelenumi{\alph{enumi})}
    \setcounter{enumi}{1}
    \tightlist
    \item
      Vitalicios.
    \end{enumerate}
  \item
    \begin{enumerate}
    \def\labelenumi{\alph{enumi})}
    \setcounter{enumi}{2}
    \tightlist
    \item
      Vitalicios Protectores.
    \end{enumerate}
  \item
    \begin{enumerate}
    \def\labelenumi{\alph{enumi})}
    \setcounter{enumi}{3}
    \tightlist
    \item
      Plenos.
    \end{enumerate}
  \item
    \begin{enumerate}
    \def\labelenumi{\alph{enumi})}
    \setcounter{enumi}{4}
    \tightlist
    \item
      Plenos ``grupo familiar''.
    \end{enumerate}
  \item
    \begin{enumerate}
    \def\labelenumi{\alph{enumi})}
    \setcounter{enumi}{5}
    \tightlist
    \item
      Cadetes.
    \end{enumerate}
  \item
    \begin{enumerate}
    \def\labelenumi{\alph{enumi})}
    \setcounter{enumi}{6}
    \tightlist
    \item
      Menores.
    \end{enumerate}
  \item
    \begin{enumerate}
    \def\labelenumi{\alph{enumi})}
    \setcounter{enumi}{7}
    \tightlist
    \item
      A distancia.
    \end{enumerate}
  \item
    \begin{enumerate}
    \def\labelenumi{\roman{enumi})}
    \tightlist
    \item
      Adherentes.
    \end{enumerate}
  \end{itemize}
\end{itemize}

\begin{itemize}
\tightlist
\item
  \textbf{Art. 10º} Podrán, ser Socios Honorarios, las personas que,
  perteneciendo o no a la Institución, se hayan hecho acreedores de esta
  distinción por haber prestado servicios de extraordinaria importancia
  a favor de la misma. El título de Socio Honorario será acordado, a
  propuesta de la Comisión Directiva o a solicitud de trescientos socios
  con derecho a voto.
\end{itemize}

\begin{itemize}
\tightlist
\item
  \textbf{Art. 11º} Serán Socios Vitalicios los asociados plenos que
  hayan cumplido una antigüedad de treinta años consecutivos como tales.
  En el último trimestre de cada año, los socios vitalicios deberán
  revalidar su condición compareciendo ante la oficina de asociados
  munidos del Documento Nacional de Identidad, Libreta de Enrolamiento o
  Libreta Cívica y fotocopia del mismo, quedando la copia archivada en
  la oficina de asociados, las que deberán ser puestas a disposición de
  la Junta Electoral cuando ese organismo lo requiera. La Comisión
  Directiva deberá intimarlos fehacientemente para que se presenten en
  la oficina de socios del club, dentro de los 45 días corridos, a
  cumplir este requisito bajo apercibimiento de excluirlos de los
  padrones electorales. Asimismo, podrán adquirir esta condición quienes
  abonen de una sola vez, como mínimo trescientas sesenta (360) cuotas
  mensuales como Socio Pleno, pudiendo ser fijado el máximo por la
  Comisión Directiva. Los Socios Vitalicios, tienen los mismos derechos
  y obligaciones que los Socios Plenos, sin el requisito de la cuota
  societaria.
\end{itemize}

\begin{itemize}
\tightlist
\item
  \textbf{Art. 12º} Los Vitalicios Protectores son aquellos que,
  habiendo obtenido la condición de vitalicio, continúen pagando el
  equivalente al 50\% de la cuota social que corresponde al socio pleno.
  La Comisión Directiva establecerá los reconocimientos y beneficios que
  se les concederán a los asociados que integren esta categoría.
\end{itemize}

\begin{itemize}
\tightlist
\item
  \textbf{Art. 13º} Serán Socios Plenos los que reúnan las siguientes
  condiciones:

  \begin{itemize}
  \item
    \begin{enumerate}
    \def\labelenumi{\alph{enumi})}
    \tightlist
    \item
      Haber cumplido 18 años de edad.
    \end{enumerate}
  \item
    \begin{enumerate}
    \def\labelenumi{\alph{enumi})}
    \setcounter{enumi}{1}
    \tightlist
    \item
      Llenar la solicitud de ingreso, acompañar su Documento Nacional de
      Identidad, Libreta de Enrolamiento o Libreta Cívica y fotocopia
      del mismo.
    \end{enumerate}
  \item
    \begin{enumerate}
    \def\labelenumi{\alph{enumi})}
    \setcounter{enumi}{2}
    \tightlist
    \item
      Abonar la cuota de ingreso vigente, el costo del carnet social,
      las mensualidades por adelantado que determine la Comisión
      Directiva y cumplir los demás requisitos que establezca la
      Comisión Directiva.
    \end{enumerate}
  \item
    \begin{enumerate}
    \def\labelenumi{\alph{enumi})}
    \setcounter{enumi}{3}
    \tightlist
    \item
      Los socios Cadetes que hubieren alcanzado la edad que requiere el
      inciso a).
    \end{enumerate}
  \item
    \begin{enumerate}
    \def\labelenumi{\alph{enumi})}
    \setcounter{enumi}{4}
    \tightlist
    \item
      Las damas, a consideración de la Comisión Directiva, podrán
      constituirse en un grupo particular de asociados, pagando una
      cuota diferenciada, con las mismas obligaciones y derechos de los
      Socios Plenos comprendidos en el presente artículo.
    \end{enumerate}
  \end{itemize}
\end{itemize}

\begin{itemize}
\tightlist
\item
  \textbf{Art. 14º} El socio Pleno ``Grupo Familiar'' será el titular y
  su cónyuge ambos con plenos derechos políticos, de una familia que se
  suscriba en todas las ramas deportivas, culturales y sociales de la
  Institución. Podrán ser adherentes de esta categoría, sus hijos
  (varones y mujeres) hasta la edad de dieciocho años (18). SE establece
  un tope de 3 hijos, y sobre el excedente de integrantes en la familia,
  la Comisión Directiva estará facultada a cobrar un extra que puede
  fijarse en un valor equivalente al 50 \% de la cuota de un socio menor
  por cada integrante que exceda en cantidad. Para ser socio ``PLENO
  GRUPO FAMILIAR'' se requiere como mínimo:

  \begin{itemize}
  \item
    \begin{enumerate}
    \def\labelenumi{\alph{enumi})}
    \tightlist
    \item
      Ser mayor de 18 años
    \end{enumerate}
  \item
    \begin{enumerate}
    \def\labelenumi{\alph{enumi})}
    \setcounter{enumi}{1}
    \tightlist
    \item
      Firmar la solicitud de ingreso y abonar por adelantado la cuota
      que fije la Comisión Directiva. En caso del fallecimiento del
      titular asume la titularidad su cónyuge. En caso de separación
      legal, el titular continúa con el mismo número de socio y
      antigüedad y el cónyuge debe asociarse nuevamente como socio pleno
      o grupo familiar según corresponda manteniendo la antigüedad.
    \end{enumerate}
  \end{itemize}
\end{itemize}

\begin{itemize}
\tightlist
\item
  \textbf{Art. 15º} Serán Socios Cadetes los que reúnan las siguiente s
  condiciones:

  \begin{itemize}
  \item
    \begin{enumerate}
    \def\labelenumi{\alph{enumi})}
    \tightlist
    \item
      Haber cumplido 12 años de edad y ser menor de 18 años;
    \end{enumerate}
  \item
    \begin{enumerate}
    \def\labelenumi{\alph{enumi})}
    \setcounter{enumi}{1}
    \tightlist
    \item
      Llenar los requisitos exigidos por los incisos b) y c) del
      \protect\hyperlink{art13}{artículo 13º} del Estatuto, acompañar
      copia de los documentos probatorios de su edad y llenar la
      solicitud de ingreso, debidamente autorizada por su representante
      legal. Los socios Cadetes al cumplir 18 años pasarán
      automáticamente a la categoría de Socio Pleno, conservando su
      antigüedad.
    \end{enumerate}
  \end{itemize}
\end{itemize}

\begin{itemize}
\tightlist
\item
  \textbf{Art. 16º} Podrá ser socio Menor, quien se suscriba en todas
  las ramas deportivas, culturales y sociales de la institución. Para
  ello se requiere como mínimo:

  \begin{itemize}
  \item
    \begin{enumerate}
    \def\labelenumi{\alph{enumi})}
    \tightlist
    \item
      Ser menor de 18 años.
    \end{enumerate}
  \item
    \begin{enumerate}
    \def\labelenumi{\alph{enumi})}
    \setcounter{enumi}{1}
    \tightlist
    \item
      Firmar la solicitud de ingreso, que será convalidada por el padre,
      tutor o encargado y abonar por adelantado la cuota que fije la
      Comisión Directiva. Los socios y socias Menores que hubieren
      alcanzado la edad de 18 años podrán ingresar a la categoría de
      socios plenos según el \protect\hyperlink{art13}{art. 13º},
      manteniendo la antigüedad. En el caso de los menores que forman
      parte del grupo familiar y cumplen los dieciocho años al hacerse
      socios plenos no se tomará en cuenta la antigüedad acumulada en el
      grupo familiar.
    \end{enumerate}
  \end{itemize}
\end{itemize}

\begin{itemize}
\tightlist
\item
  \textbf{Art. 17º} Podrán ser socios ``a distancia'' todo aquel
  simpatizante que viviendo fuera del radio urbano de la Ciudad de
  Córdoba llene los requisitos que establezca la Comisión Directiva y
  cumpla con los requisitos del \protect\hyperlink{art13}{artículo 13º}
  . La cuota de los socios a distancia se regirá por los
  \protect\hyperlink{art68}{artículos 68º} y
  \protect\hyperlink{art123}{123º} del presente Estatuto.
\end{itemize}

\begin{itemize}
\tightlist
\item
  \textbf{Art. 18º} Los socios adherentes serán considerados como
  simpatizantes de la Institución y carecen de derechos políticos; no
  pudiendo ingresar a la Sede ni participar de la vida social de la
  Institución; podrán presenciar los partidos de fútbol del equipo de
  primera división que se disputen en el estadio de la Institución, o en
  el que nuestra entidad haga las veces de local, con acceso libre a la
  tribuna popular local, siempre y cuando tenga el pago de la cuota
  social que corresponda a esta categoría al día.
\end{itemize}

\begin{itemize}
\tightlist
\item
  \textbf{Art. 19º} La Comisión Directiva podrá rechazar por simple
  mayoría y sin apelaciones, las solicitudes de ingreso. A su vez se
  reserva el derecho de admisión.
\end{itemize}

\begin{itemize}
\tightlist
\item
  \textbf{Art. 20º} Los asociados pierden el carácter de tal por
  renuncia, expulsión o exclusión. Las causas de bajas son:

  \begin{itemize}
  \item
    \begin{enumerate}
    \def\labelenumi{\alph{enumi})}
    \tightlist
    \item
      renuncia, los que estando al día con tesorería, lo soliciten por
      escrito a la Comisión Directiva.
    \end{enumerate}
  \item
    \begin{enumerate}
    \def\labelenumi{\alph{enumi})}
    \setcounter{enumi}{1}
    \tightlist
    \item
      expulsión, se aplicará a los socios incursos en las siguientes
      faltas; Primero: incumplimiento reiterado de las obligaciones
      impuestas por el estatuto y los reglamentos internos; Segundo:
      hacer daño al Club y observar conducta notoriamente perjudicial a
      los intereses sociales;
    \end{enumerate}
  \item
    \begin{enumerate}
    \def\labelenumi{\alph{enumi})}
    \setcounter{enumi}{2}
    \tightlist
    \item
      exclusión, se aplicará automáticamente a los que adeuden cuatro
      meses consecutivos de cuotas.
    \end{enumerate}
  \end{itemize}
\end{itemize}

\begin{itemize}
\tightlist
\item
  \textbf{Art. 21º} Las sanciones previstas a los asociados, así como
  las de suspensión o amonestación, serán aplicadas por la Comisión
  Directiva, la que previo a ello deberá intimar al imputado a
  comparecer a una reunión de dicha Comisión, en la fecha y hora que se
  le comunicara mediante notificación fehaciente, cursada con una
  anticipación mínima de diez días corridos y conteniendo la enunciación
  del hecho punible y de la norma presumiblemente transgredida, como así
  la invitación a realizar descargos, ofrecer prueba y alegar sobre la
  producida. La no comparecencia del interesado implica la renuncia al
  ejercicio del derecho de defensa y la presunción de verosimilitud de
  los cargos formulados, quedando la Comisión Directiva habilitada para
  resolver.
\end{itemize}

\begin{itemize}
\tightlist
\item
  \textbf{Art. 22º} De todas las resoluciones adoptadas en su contra por
  la Comisión Directiva, el asociado, podrá apelar en primera instancia
  a la Junta Representativa y en segunda instancia, ante la primera
  Asamblea que se realice, siempre que se presente el respectivo
  recurso, en forma escrita ante la Comisión Directiva, dentro de los
  diez días de notificado de su sanción.
\end{itemize}

\begin{itemize}
\tightlist
\item
  \textbf{Art. 23º} Los socios dados de baja por renuncia o exclusión,
  solo podrán ingresar al Club, como socios nuevos, perdiendo la
  antigüedad anterior.
\end{itemize}

\begin{itemize}
\tightlist
\item
  \textbf{Art. 24º} Mientras no sea excluido, el socio moroso podrá
  ponerse al día, conservando así su antigüedad, pagando la totalidad de
  las cuotas adeudadas a valores actuales.
\end{itemize}

\chapter{DE LOS DEBERES Y OBLIGACIONES DE LOS
ASOCIADOS}\label{de-los-deberes-y-obligaciones-de-los-asociados}

\begin{itemize}
\item
  \textbf{Art. 25º} Son obligaciones de los asociados:

  \begin{itemize}
  \item
    \begin{enumerate}
    \def\labelenumi{\alph{enumi})}
    \tightlist
    \item
      Conceder, respetar y cumplir las disposiciones de este Estatuto,
      reglamentos y resoluciones de la Asamblea y Comisión Directiva,
      así como mantener el orden y decoro dentro y fuera de las
      dependencias del Club y en todos los actos que se intervenga.
    \end{enumerate}
  \item
    \begin{enumerate}
    \def\labelenumi{\alph{enumi})}
    \setcounter{enumi}{1}
    \tightlist
    \item
      Abonar mensualmente y por adelantado las cuotas sociales.
    \end{enumerate}
  \item
    \begin{enumerate}
    \def\labelenumi{\alph{enumi})}
    \setcounter{enumi}{2}
    \tightlist
    \item
      Tener la obligación moral de aceptar las Comisiones que les
      encomiendan las Asambleas y Comisiones Directivas.
    \end{enumerate}
  \item
    \begin{enumerate}
    \def\labelenumi{\alph{enumi})}
    \setcounter{enumi}{3}
    \tightlist
    \item
      Comunicar, dentro de los diez días corridos, todo cambio de
      domicilio a la Comisión Directiva.
    \end{enumerate}
  \end{itemize}
\item
  \textbf{Art. 26º} Las cuotas mensuales de cada categoría de socios,
  serán fijadas por la Comisión Directiva. Los asociados admitidos en la
  segunda quincena, pagarán la mensualidad íntegra, lo mismo ocurrirá
  con los que dejen de pertenecer al club en la primera quincena. El
  pago de las cuotas sociales vence el 10 de cada mes.
\item
  \textbf{Art. 27º} Están eximidos del pago de las cuotas, los socios
  comprendidos en la siguiente situación: solicitar por escrito a la
  Comisión Directiva, una licencia hasta un plazo máximo de seis meses y
  siempre que la causa invocada se justifique ampliamente. Durante la
  licencia, el asociado, no podrá concurrir al local social sin razón
  atendible, pues su presencia en el Club, significará la reanudación de
  sus obligaciones para con la Institución.
\item
  \textbf{Art. 28º} Los asociados acreditarán su condición de tal,
  exhibiendo el carnet social con fotografía y el recibo deberá estar al
  día con Tesorería, cuya presentación es obligatoria para el ingreso en
  las dependencias del Club y el uso de distintos servicios sociales. El
  carnet social y los recibos de cuotas son personales e
  intransferibles.
\item
  \textbf{Art. 29º} Los socios son responsables de cualquier deterioro o
  perjuicio material que pudieran ocasionar a la Institución,
  responsabilidad que se extiende a los padres, tutores o guardadores de
  los socios menores de 18 años de edad.
\item
  \textbf{Art. 30º} Los socios podrán ser suspendidos en sus derechos ,
  por las siguientes causas:

  \begin{itemize}
  \item
    \begin{enumerate}
    \def\labelenumi{\alph{enumi})}
    \tightlist
    \item
      Por no respetar el Estatuto Social, los Reglamentos internos o las
      resoluciones dictadas por las Asambleas o la Comisión Directiva.
    \end{enumerate}
  \item
    \begin{enumerate}
    \def\labelenumi{\alph{enumi})}
    \setcounter{enumi}{1}
    \tightlist
    \item
      Por no mantener el orden y el decoro dentro y fuera de las
      instalaciones del Club, en todos los actos en que se intervenga.
    \end{enumerate}
  \item
    \begin{enumerate}
    \def\labelenumi{\alph{enumi})}
    \setcounter{enumi}{2}
    \tightlist
    \item
      Por ceder o facilitar a otras personas el carnet social. Cuando un
      socio fuera sorprendido faltando al cumplimiento de sus
      obligaciones, según las prescripciones de los incisos a), b) y c),
      cualquier miembro de la Comisión Directiva o subcomisión, o en
      ausencia de esta, el Gerente, o Intendente, podrán resolver el
      retiro del carnet, elevando el informe pertinente a la Comisión
      Directiva dentro de las 48 horas. La negativa del socio a entregar
      la referida credencial, cuando le sea requerida, implicará
      desacato y agravará su falta. La suspensión podrá ser de hasta
      seis meses de acuerdo a la gravedad del hecho constatado. El que
      fuera suspendido por dos veces, a la tercera sanción, será
      expulsado. El socio abonará igualmente las cuo tas sociales
      correspondientes al período de suspensión.
    \end{enumerate}
  \end{itemize}
\item
  \textbf{Art. 31º} La ignorancia de este Estatuto no servirá de excusa
  si la excepción no está expresamente autorizada en el mismo, por lo
  tanto, desde el instante de su ingreso como asociado, declara conocer
  este Estatuto y se compromete a cumplirlo fielmente.
\end{itemize}

\chapter{DE LOS DERECHOS DE LOS
ASOCIADOS}\label{de-los-derechos-de-los-asociados}

\begin{itemize}
\item
  \textbf{Art. 32º} Los asociados, por el hecho de serlo, hallándose al
  día en el pago de las cuotas sociales, y no estando comprendidos en el
  \protect\hyperlink{art20}{artículo 20º} tienen derecho:

  \begin{itemize}
  \item
    \begin{enumerate}
    \def\labelenumi{\alph{enumi})}
    \tightlist
    \item
      de voto, siempre que tengan más de 18 años de edad con un año de
      antigüedad ininterrumpido como socio de la Institución. Ambos
      requisitos se cumplirán a la fecha de las elecciones.
    \end{enumerate}
  \item
    \begin{enumerate}
    \def\labelenumi{\alph{enumi})}
    \setcounter{enumi}{1}
    \tightlist
    \item
      De ser elegidos, para los distintos cargos de la Institución, bajo
      las condiciones estatutarias de cada cargo.
    \end{enumerate}
  \item
    \begin{enumerate}
    \def\labelenumi{\alph{enumi})}
    \setcounter{enumi}{2}
    \tightlist
    \item
      A voz y voto en las asambleas, exceptuándose aquellos que no
      tuvieran una antigüedad de un año.
    \end{enumerate}
  \item
    \begin{enumerate}
    \def\labelenumi{\alph{enumi})}
    \setcounter{enumi}{3}
    \tightlist
    \item
      Integrar las distintas subcomisiones.
    \end{enumerate}
  \item
    \begin{enumerate}
    \def\labelenumi{\alph{enumi})}
    \setcounter{enumi}{4}
    \tightlist
    \item
      Usar las instalaciones sociales y gozar de todos los beneficios
      que le acuerde el Estatuto y reglamentos en vigor.
    \end{enumerate}
  \item
    \begin{enumerate}
    \def\labelenumi{\alph{enumi})}
    \setcounter{enumi}{5}
    \tightlist
    \item
      Al acceso al estadio de acuerdo con los reglamentos vigentes o
      resoluciones que pudiere dictar en cada caso la Comisión
      Directiva.
    \end{enumerate}
  \item
    \begin{enumerate}
    \def\labelenumi{\alph{enumi})}
    \setcounter{enumi}{6}
    \tightlist
    \item
      A practicar todos los deportes y demás juegos existentes en las
      dependencias de la Institución, con arreglo y sujeción a las
      condiciones y reglamentaciones especiales que por resolución de la
      Comisión Directiva se establecieren al respecto.
    \end{enumerate}
  \item
    \begin{enumerate}
    \def\labelenumi{\alph{enumi})}
    \setcounter{enumi}{7}
    \tightlist
    \item
      A traer visitantes a las dependencias de la Institución, con el
      objeto de hacer conocer las mismas de acuerdo a la respectiva
      reglamentación.
    \end{enumerate}
  \item
    \begin{enumerate}
    \def\labelenumi{\roman{enumi})}
    \tightlist
    \item
      Concurrir a todos los espectáculos deportivos, sociales y
      culturales, previo pago de las cuotas adicionales y entradas que
      fije la Comisión Directiva, cuando ésta lo estimare conveniente.
    \end{enumerate}
  \item
    \begin{enumerate}
    \def\labelenumi{\alph{enumi})}
    \setcounter{enumi}{9}
    \tightlist
    \item
      Peticionar ante la Comisión Directiva la citación a Asamblea
      Extraordinaria de Socios, con nota fundada que deberá ser firmada
      por asociados con derecho a voto y que representan, por lo menos,
      el 10\% del total de los asociados inscriptos que reúnan ese
      requisito a la fecha de la presentación de la misma.
    \end{enumerate}
  \item
    \begin{enumerate}
    \def\labelenumi{\alph{enumi})}
    \setcounter{enumi}{10}
    \tightlist
    \item
      Solicitar licencias.
    \end{enumerate}
  \item
    \begin{enumerate}
    \def\labelenumi{\alph{enumi})}
    \setcounter{enumi}{11}
    \tightlist
    \item
      Impugnar la presentación y admisión de socios, formulando las
      observaciones del caso.
    \end{enumerate}
  \item
    \begin{enumerate}
    \def\labelenumi{\alph{enumi})}
    \setcounter{enumi}{12}
    \tightlist
    \item
      A revocar el mandato de toda la Comisión Directiva, Junta
      Representativa y Comisión Revisora de Cuentas transcurrido un año
      de su gestión y por considerar su incapacidad para continuar
      conduciendo la Institución, según lo establecido en el
      \protect\hyperlink{art150}{artículo 150º}.
    \end{enumerate}
  \item
    \begin{enumerate}
    \def\labelenumi{\alph{enumi})}
    \setcounter{enumi}{13}
    \tightlist
    \item
      Presentar asociados.
    \end{enumerate}
  \end{itemize}
\item
  \textbf{Art. 33º} Los socios honorarios tendrán los mismos derechos ,
  deberes y obligaciones que este Estatuto acuerda a los asociados
  plenos, vitalicios, vitalicios protectores y especiales con excepción
  de los enunciados en el inciso b) del
  \protect\hyperlink{art25}{artículo 25º} y en los incisos a), b), j) y
  m) del \protect\hyperlink{art32}{artículo 32º}. Estas limitaciones no
  rigen para los asociados honorarios que tuvieren derecho a su
  ejercicio por revistar a la vez en las categorías de vitalicios,
  protectores o plenos.
\item
  \textbf{Art. 34º} Los socios Vitalicios quedan exentos del pago de la
  cuota mensual, teniendo en lo demás los mismos derechos, deberes y
  obligaciones que los socios activos. Igual tratamiento tendrán los
  Vitalicios Protectores excepto por su compromiso a abonar una cuota
  mensual, en correspondencia con el \protect\hyperlink{art12}{artículo
  12º}.
\item
  \textbf{Art. 35º} Los empleados a sueldo de la Institución o que
  perciban cualquier tipo de honorario o remuneración podrán ser socios
  y tendrá n en tal supuesto los derechos, deberes y obligaciones de
  éstos; menos de intervenir en política electoral de la Institución, ni
  ser electores ni ser elegidos o el de peticionar ante las autoridades,
  hasta después de un año de haber cesado en su empleo o de sus
  funciones remuneradas.
\end{itemize}

\chapter{DE LAS AUTORIDADES Y FISCALIZACIÓN DE LA
ASOCIACIÓN}\label{de-las-autoridades-y-fiscalizacion-de-la-asociacion}

\begin{itemize}
\tightlist
\item
  \textbf{Art. 36º} Son autoridades del Club Instituto Atlético Central
  Córdoba y como tales deben ser reconocidas y respetadas por los
  socios, las que a continuación se mencionan:

  \begin{itemize}
  \item
    \begin{enumerate}
    \def\labelenumi{\alph{enumi})}
    \tightlist
    \item
      La Asamblea constituida estatutariamente.
    \end{enumerate}
  \item
    \begin{enumerate}
    \def\labelenumi{\alph{enumi})}
    \setcounter{enumi}{1}
    \tightlist
    \item
      La Comisión Directiva y sus miembros.
    \end{enumerate}
  \item
    \begin{enumerate}
    \def\labelenumi{\alph{enumi})}
    \setcounter{enumi}{2}
    \tightlist
    \item
      La Junta Representativa y sus miembros.
    \end{enumerate}
  \item
    \begin{enumerate}
    \def\labelenumi{\alph{enumi})}
    \setcounter{enumi}{3}
    \tightlist
    \item
      La Comisión Revisora de Cuentas y sus miembros.
    \end{enumerate}
  \item
    \begin{enumerate}
    \def\labelenumi{\alph{enumi})}
    \setcounter{enumi}{4}
    \tightlist
    \item
      La Junta Electoral y sus miembros, mientras estén en funciones.
    \end{enumerate}
  \item
    \begin{enumerate}
    \def\labelenumi{\alph{enumi})}
    \setcounter{enumi}{5}
    \tightlist
    \item
      Los socios que integran las Subcomisiones autorizadas por la
      Comisión Directiva.
    \end{enumerate}
  \end{itemize}
\end{itemize}

\begin{itemize}
\item
  \textbf{Art. 37º} Cada órgano es juzgador de sus propios miembros, a
  quienes puede amonestar o suspender en su carácter de tales, por
  causas que repute graves. Para la destitución convocará a Asamblea de
  Socios, por el voto de la mitad más uno de sus integrantes. La
  destitución de cualquier autoridad del Club es de competencia
  exclusiva de la Asamblea de Socios y deberá contar con el voto
  afirmativo de las 2/3 partes del total de los miembros de la misma. El
  pronunciamiento de este órgano puede limitarse a la destitución en el
  cargo electivo o extenderse a la expulsión del afectado en su
  condición de socio.
\item
  \textbf{Art. 38º} Los miembros de Comisión Directiva, Junta
  Representativa, Comisión Revisora de Cuentas, y Subcomisiones
  autorizadas por Comisión Directiva, contra quienes sus propios cuerpos
  pidan la destitución del cargo, quedarán suspendidos en el desempeño
  del mismo hasta tanto se pronuncie la Asamblea de Socios.
\item
  \textbf{Art. 39º} Todo miembro de un órgano puede salvar su
  responsabilidad respecto de las resoluciones de su cuerpo, dejando
  expresa constancia en las actas pertinentes de su votación en contra.
\end{itemize}

\chapter{DE LAS ASAMBLEAS}\label{de-las-asambleas}

\begin{itemize}
\item
  \textbf{Art. 40º} Las Asambleas serán Ordinarias y Extraordinarias,
  las cuales todas ellas serán soberanas. Las Asambleas Ordinarias se
  celebrarán anualmente dentro de los 90 días corridos del cierre del
  ejercicio, y en ellas se incluirá, sin excepción, el siguiente Orden
  del Día:

  \begin{itemize}
  \item
    \begin{enumerate}
    \def\labelenumi{\alph{enumi})}
    \tightlist
    \item
      Designación de dos señores socios para firmar el acta.
    \end{enumerate}
  \item
    \begin{enumerate}
    \def\labelenumi{\alph{enumi})}
    \setcounter{enumi}{1}
    \tightlist
    \item
      Lectura y consideración del acta de la Asamblea anterior.
    \end{enumerate}
  \item
    \begin{enumerate}
    \def\labelenumi{\alph{enumi})}
    \setcounter{enumi}{2}
    \tightlist
    \item
      Consideración de la memoria, Balance General e Informe de la
      Comisión Revisora de Cuentas, por el ejercicio anual cerrado el 30
      de Junio.
    \end{enumerate}
  \item
    \begin{enumerate}
    \def\labelenumi{\alph{enumi})}
    \setcounter{enumi}{3}
    \tightlist
    \item
      Consideración de cualquier otro tema incluido en el Orden del día
      de la Convocatoria. La Asamblea Extraordinaria deberá ser
      convocada, cuando lo sea a petición de asociados, dentro de los
      treinta (30) días corridos de formulada la solicitud, o cuando la
      Comisión Directiva lo creyere conveniente, en cuyo caso efectuará
      la convocatoria con treinta (30) días corridos de anticipación a
      la fecha de realización de la misma. Cuando fuera a petición de
      asociados, la solicitud deberá formularse por escrito y estar
      suscripta por lo menos por el diez por ciento (10\%) de los socios
      con derecho a voto y con indicación expresa de los motivos del
      requerimiento.
    \end{enumerate}
  \end{itemize}
\item
  \textbf{Art. 41º} La Comisión Directiva convocará, ineludiblemente, a
  la Asamblea General Extraordinaria para el tratamiento de la compra,
  venta o gravamen de bienes inmuebles y para el reemplazo, modificación
  o corrección del Estatuto, disolución o fusión de la entidad con otra.
\item
  \textbf{Art. 42º} En las Asambleas no podrán tratarse asuntos no
  incluidos en la Orden del Día, salvo cuestiones vinculadas a aquellos
  o a la validez de la convocatoria.
\end{itemize}

\begin{itemize}
\item
  \textbf{Art. 43º} Las Asambleas se constituirán, en primera
  convocatoria, en el día, hora y lugar indicados en la misma, con un
  número igual a la mitad más uno de los socios con derecho a voz y
  voto. Si no hubiere número en l a primera citación, media hora después
  de la fijada para la primera, con cualquier número de socios siempre
  que no sea inferior al total de miembros titulares que componen la
  Comisión Directiva, haciendo exclusión -para el recuento- de éstos.
\item
  \textbf{Art. 44º} La convocatoria, con el Orden del Día confeccionado,
  deberá hacerse conocer a los asociados por el Boletín Oficial durante
  el término de tres (3) días, y paralelamente a ello, por
  notificaciones expuestas en las carteleras de la Sede Social, del
  Estadio y del Predio La Agustina. Todo ello con treinta (30) días
  corridos de anticipación al acto
\item
  \textbf{Art. 45º} Los socios Vitalicios, Vitalicios Protectores,
  Plenos y los titulares de la categoría Grupo Familiar, sin distinción
  de sexo, que estuvieren al corriente con Tesorería, podrán participar
  de las asambleas con derecho a voz y voto, siempre que no se
  encontraren en uso de licencia o cumpliendo sanciones disciplinarias.
  A los socios Plenos y titulares de la categoría Grupo Familiar se les
  exigirá una antigüedad mínima consecutiva e inmediata de un (1) año en
  el carácter de tales. A los socios que hubieran cambiado de categoría,
  antes de formulada la convocatoria, se les computará el tiempo que
  hubiesen pertenecido a otras categorías, siempre que fuese continuo e
  inmediato.
\item
  \textbf{Art. 46º} La participación de los socios en las Asambleas será
  directa y personal, sin que nadie pueda hacerse representar en modo y
  forma alguna ante ellas, ni retirarse de las mismas sin dar aviso a la
  Presidencia. Si por el retiro de socios, la Asamblea quedase sin el
  número mínimo previsto en el \protect\hyperlink{art43}{artículo 43º},
  así lo declarará la Presidencia dándola por terminada. Que dando algún
  asunto pendiente, se hará una nueva convocatoria, en la forma y el
  plazo determinados en este Estatuto.
\item
  \textbf{Art. 47º} Las asambleas serán presididas por el Presidente o
  Vice-Presidente 1º; a falta de éstos, por el Vice-Presidente 2º,
  Secretario, Pro-Secretario o Tesorero y por ausencia de los anteriores
  citados, será presidida por el socio que la misma Asamblea designare.
\item
  \textbf{Art. 48º} Ningún socio podrá hacer uso de la palabra sin que
  ella le fuere concedida por el Presidente, quien deberá acordarla
  según el siguiente orden:

  \begin{itemize}
  \item
    \begin{enumerate}
    \def\labelenumi{\alph{enumi})}
    \tightlist
    \item
      Al miembro informante de la Comisión Directiva.
    \end{enumerate}
  \item
    \begin{enumerate}
    \def\labelenumi{\alph{enumi})}
    \setcounter{enumi}{1}
    \tightlist
    \item
      Al miembro informante en disidencia.
    \end{enumerate}
  \item
    \begin{enumerate}
    \def\labelenumi{\alph{enumi})}
    \setcounter{enumi}{2}
    \tightlist
    \item
      Al autor del proyecto.
    \end{enumerate}
  \item
    \begin{enumerate}
    \def\labelenumi{\alph{enumi})}
    \setcounter{enumi}{3}
    \tightlist
    \item
      Al primero que la pidiese entre los demás socios. El Presidente
      podrá retirar el uso de la palabra al socio que se personalice con
      otro, al que se aparte del asunto en discusión, lo hiciere en
      términos inconvenientes o violatorios del Estatuto; deberá
      suspender diálogos e interrupciones y si se insistiere en los
      actos precedentemente enumerados podrá hacerlo retirar del
      recinto, previa votación de la Asamblea en ese sentido.
    \end{enumerate}
  \end{itemize}
\item
  \textbf{Art. 49º} El Presidente podrá cerrar el debate, siempre que la
  Asamblea no disponga lo contrario a moción de cualquier socio.
\item
  \textbf{Art. 50º} Las votaciones podrán ser públicas, secretas,
  nominales o por signos, según determinare el Presidente en cada caso.
\item
  \textbf{Art. 51º} Las resoluciones se adoptarán por simple mayoría de
  los socios presentes, salvo los casos previstos por este estatuto que
  exige proporción mayor. Ningún socio podrá tener más de un voto y los
  miembros de la Comisión Directiva y Comisión Revisora de cuentas; se
  abstendrán de hacerlo en los asuntos relacionados con su gestión. Para
  reconsiderar resoluciones adoptadas en Asambleas anteriores, se podrá
  hacerlo solamente un a vez, y se requerirá el voto favorable de los
  dos tercios de los socios presentes en otra Asamblea constituida corno
  mínimo, con igual o mayor número de asistentes, al de aquella que
  resolvió el asunto a reconsiderar.
\item
  \textbf{Art. 52º} Para ser aprobados los puntos que se refieren a la
  venta y/o gravamen de bienes inmuebles, incorporados al patrimonio del
  Club, como así también la fusión con otra u otras instituciones
  similares o no, se requerirá el voto favorable de los dos tercios
  (2/3) de los socios presentes en la Asamblea Extraordinaria
  constituida con un quórum mínimo del diez por ciento (10\%) de los
  asociados con derecho a voto.
\item
  \textbf{Art. 53º} Para la destitución de alguna de las autoridades
  sancionadas por sus respectivos órganos de gobierno, en virtud del
  \protect\hyperlink{art37}{artículo 37º} del presente Estatuto Social,
  se requerirán los dos tercios (2/3) de los votos de los socios
  presentes en la Asamblea, sea Ordinaria o Extraordinaria, constituida
  con un quórum mínimo del diez por ciento (10\%) de los asociados con
  derecho a voto.
\item
  \textbf{Art. 54º} La reforma o cambio de estos Estatutos requerirá la
  constitución de una Asamblea Extraordinaria, con un quórum mínimo del
  diez por ciento (10\%) de los socios con derecho a voto. La resolución
  se tomara por simple mayoría. El proyecto de reforma deberá estar a
  disposición de los asociados, en la Sede Social, con quince (15) días
  corridos de anticipación a la Asamblea.
\end{itemize}

\begin{itemize}
\item
  \textbf{Art. 55º} Las Asambleas podrán pasar a cuarto intermedio por
  un término que no excederá, en ningún caso, los siete (7) días
  corridos, a cuyo vencimiento deberán --indefectiblemente- continuar.
\item
  \textbf{Art. 56º} Las asambleas designarán dos miembros de su seno
  para firmar el acta juntamente con el Presidente y Secretario.
\end{itemize}

\chapter{DE LA COMISIÓN DIRECTIVA}\label{de-la-comision-directiva}

\begin{itemize}
\tightlist
\item
  \textbf{Art. 57º} La Institución será administrada, dirigida y
  representada en todos sus actos y contratos por una Comisión
  Directiva, -- ad honorem -, compuesta de la siguiente manera: un
  Presidente, un Vicepresidente 1º, un Vicepresidente 2º, un Secretario
  General, un Prosecretario General, un Tesorero y un Protesorero; un
  Secretario de Actas e Información Institucional, un Prosecretario de
  Actas e Información Institucional, un Secretario de Recursos
  Financieros, un Prosecretario de Recursos Financieros y cinco vocales.
\end{itemize}

\begin{itemize}
\tightlist
\item
  \textbf{Art. 58º} Para ser miembro de la Comisión Directiva, se
  requiere:

  \begin{itemize}
  \item
    \begin{enumerate}
    \def\labelenumi{\alph{enumi})}
    \tightlist
    \item
      Haber cumplido 25 años de edad y cuatro (4) consecutivos e
      inmediatos de antigüedad como socio Pleno, Vitalicio o Vitalicio
      Protector, en cualquiera de esas categorías o entre ellas. Podrá
      haber dos miembros menores de veinticinco años pero mayores de
      dieciocho, también con (4) consecutivos e inmediatos de
      antigüedad, por lo menos, como socios Plenos.
    \end{enumerate}
  \item
    \begin{enumerate}
    \def\labelenumi{\alph{enumi})}
    \setcounter{enumi}{1}
    \tightlist
    \item
      Estar al día con la Tesorería.
    \end{enumerate}
  \item
    \begin{enumerate}
    \def\labelenumi{\alph{enumi})}
    \setcounter{enumi}{2}
    \tightlist
    \item
      No haber sido condenado por la justicia a pena infamante, si
      hubiera sido concursado o quebrado, deberá ser rehabilitado
      judicialmente; d) No ser empleado del Club.
    \end{enumerate}
  \item
    \begin{enumerate}
    \def\labelenumi{\alph{enumi})}
    \setcounter{enumi}{3}
    \tightlist
    \item
      No pertenecer a otros órganos de gobierno de instituciones
      similares, ni ser empleado de ellas o que practique deporte de
      carácter profesional.
    \end{enumerate}
  \end{itemize}
\end{itemize}

\begin{itemize}
\tightlist
\item
  \textbf{Art. 59º} Los miembros de la Comisión Directiva durarán tres
  ejercicios en sus cargos. Los cargos de Presidente, Vicepresidente 1º,
  Vicepresidente 2º, Secretario General y Tesorero serán conferidos en
  elección directa en los comicios de asociados. En cuanto a los cargos
  de Prosecretario General, Protesorero, Secretario de Actas e
  Información Institucional, Prosecretario de Actas e Información
  Institucional, Secretario de Recursos Financieros, y Prosecretario de
  Recursos Financieros, serán designados por la mayoría de la totalidad
  de los miembros de Comisión Directiva en la primera reunión que se
  celebre después de los comicios. La elección de Comisión Directiva se
  hará por lista completa que contendrá los candidatos a Presidente,
  Vicepresidente 1º, Vicepresidente 2º, Secretario General, Tesorero y
  once (11) candidatos a Vocales, por orden de lista.
\end{itemize}

\begin{itemize}
\tightlist
\item
  \textbf{Art. 60º} Los miembros de Comisión Directiva durarán tres (3)
  ejercicios en su mandato y podrán ser reelegidos, a excepción del
  Presidente que podrá ser reelegido por solamente un período. Si ha
  sido reelecto no puede ser elegido para el mismo cargo, sino con el
  intervalo de un período.
\end{itemize}

\begin{itemize}
\tightlist
\item
  \textbf{Art. 61º} La Comisión Directiva tendrá la obligación de
  reunirse en sesión ordinaria, por lo menos, 2 veces al mes y en sesión
  extraordinaria cuando lo juzgue necesario, bastando para su
  convocatoria, la simple resolución de la presidencia o la petición
  escrita de seis (6) de sus miembros titulares. Las sesiones serán
  públicas, salvo en casos que la Comisión Directiva resuelva lo
  contrario por circunstancias especiales de su propio juzgamiento.
\end{itemize}

\begin{itemize}
\tightlist
\item
  \textbf{Art. 62º} La Comisión Directiva, en cualquiera de los casos
  previstos, podrá deliberar con la presencia de la mitad más uno de sus
  miembros, lo que hará constar con la firma del libro de asistencia que
  se llevará por secretaría.
\end{itemize}

\begin{itemize}
\tightlist
\item
  \textbf{Art. 63º} Todas las resoluciones de la Comisión Directiva,
  salvo disposición estatutaria en contrario, requerirán para su
  aprobación el voto favorable de la mitad más uno de los Miembros
  Titulares presentes.
\end{itemize}

\begin{itemize}
\tightlist
\item
  \textbf{Art. 64º} Las reuniones de la Comisión Directiva serán
  presididas por el Presidente. A falta de éste, lo será por el miembro
  de Comisión Directiva presente que revista mayor rango y que se
  designara a tal efecto, según el orden de prelación establecido por el
  \protect\hyperlink{art57}{Artículo 57º}.
\end{itemize}

\begin{itemize}
\tightlist
\item
  \textbf{Art. 65º} Ninguna resolución de la Comisión Directiva podrá
  ser reconsiderada sino por la mayoría de dos tercios (2/3) de votos,
  en otra sesión que cuente con igual o mayor número de miembros
  presentes, que aquella en la que hubiera sido votada la resolución que
  se pretende reconsiderar. En ese caso, el quórum deberá formarse con
  un número igual o mayor de titulares presentes en la sesión en que
  aquella se tomó. Para ser votada una moción de reconsideración, será
  necesaria que la apoyen cuatro (4) miembros titulares presentes.
\end{itemize}

\begin{itemize}
\tightlist
\item
  \textbf{Art. 66º} Cuando un miembro de la Comisión Directiva faltase a
  cinco (5) sesiones consecutivas sin causa justificada, u ocho (8)
  alternadas, deberá ser prevenido por escrito y en caso de nueva
  ausencia similar inmediata, será declarado cesante, incorporándose
  automáticamente al miembro de Junta Representativa de mayor rango en
  forma definitiva.
\end{itemize}

\begin{itemize}
\tightlist
\item
  \textbf{Art. 67º}

  \begin{itemize}
  \item
    \begin{enumerate}
    \def\labelenumi{\alph{enumi})}
    \tightlist
    \item
      Si se produjera la acefalía de la Comisión Directiva, ya fuera por
      renuncia colectiva de sus miembros o por haber quedado en minoría
      absoluta, situación que se presenta cuando por diversas causas la
      cantidad de integrantes de la misma ha quedado reducida a menos de
      la mitad del total de socios que la componen, se hará cargo de la
      entidad, automáticamente, la Junta Representativa, constituida de
      conformidad con lo establecido por el
      \protect\hyperlink{art105}{artículo 105º}. La mencionada Junta
      actuará como Comisión Directiva del Club, con todos sus deberes y
      atribuciones, por un lapso máximo de sesenta (60) días corridos.
      Dentro de los primeros veinte (20) días corridos de iniciadas sus
      funciones, deberá convocar a elecciones para conformar una nueva
      Comisión Directiva.
    \end{enumerate}
  \item
    \begin{enumerate}
    \def\labelenumi{\alph{enumi})}
    \setcounter{enumi}{1}
    \tightlist
    \item
      Luego que de acuerdo con los \protect\hyperlink{art146}{artículos
      146º} y \protect\hyperlink{art148}{148º} la Junta Electoral
      informe sobre el resultado de las elecciones, los asociados
      electos recibirán de la Junta Representativa la dirección del
      Club, en reunión conjunta, lo que se hará constar en acta.
    \end{enumerate}
  \item
    \begin{enumerate}
    \def\labelenumi{\alph{enumi})}
    \setcounter{enumi}{2}
    \tightlist
    \item
      La Comisión Directiva elegida con motivo de la eventualidad
      prevista en el apartado a) asumirá sus funciones por el periodo
      estatutariamente previsto.
    \end{enumerate}
  \item
    \begin{enumerate}
    \def\labelenumi{\alph{enumi})}
    \setcounter{enumi}{3}
    \tightlist
    \item
      En caso de acefalía de Comisión Directiva y Junta Representativa,
      la Comisión Revisora de Cuentas actuará como Comisión Directiva
      del Club, con todos los deberes y atribuciones previstos en el
      inciso a) del presente artículo.
    \end{enumerate}
  \end{itemize}
\end{itemize}

\begin{itemize}
\tightlist
\item
  \textbf{Art. 68º} Son facultades, deberes y atribuciones de la
  Comisión Directiva según los casos y sin perjuicio de lo que
  tácitamente resulte del presente Estatuto:

  \begin{itemize}
  \item
    \begin{enumerate}
    \def\labelenumi{\alph{enumi})}
    \tightlist
    \item
      Respetar y hacer respetar el Estatuto, Reglamentos y resoluciones
      que en uso de sus facultades, dicten las autoridades de la
      Asociación del Fútbol Argentino, absteniéndose de efectuar por si
      y/o por medio de sus representantes protestas públicas contra
      aquellas y/o cuestionarlas, salvo casos de arbitrariedad por
      ilegitimidad o nulidad por violación de las formas esenciales del
      procedimiento, en un todo de acuerdo con lo establecido en el art.
      6º del Estatuto de la Asociación del Fútbol Argentino.
    \end{enumerate}
  \item
    \begin{enumerate}
    \def\labelenumi{\alph{enumi})}
    \setcounter{enumi}{1}
    \tightlist
    \item
      Procurar y velar por el mejor cumplimiento del presente Estatuto,
      reglamentos internos y resoluciones vigentes, pudiendo resolver
      por si, todo lo que a ello no se oponga y siempre que no se
      afecten los propósitos básicos e intereses de la Institución.
    \end{enumerate}
  \item
    \begin{enumerate}
    \def\labelenumi{\alph{enumi})}
    \setcounter{enumi}{2}
    \tightlist
    \item
      Ejercer el contralor directo de los fondos del Club, sin que estos
      puedan ser distraídos o aplicados, bajo ningún concepto, para
      otros fines que no fueran de interés o beneficio para la
      Institución, sus socios o la comunidad.
    \end{enumerate}
  \item
    \begin{enumerate}
    \def\labelenumi{\alph{enumi})}
    \setcounter{enumi}{3}
    \tightlist
    \item
      Convocar anualmente a la Asamblea General Ordinaria para
      considerar los asuntos que le fueren propios y en particular, la
      Memoria y Balance anuales, que se presentarán a la consideración
      de los asociados, con el informe de la Comisión Revisora de
      Cuentas y de los actos y hechos más importantes ocurridos en el
      ejercicio, como así también el presupuesto anual de gastos y
      recursos, que deberá someter a control permanente de la Asociación
      del Fútbol Argentino, en un todo de acuerdo con lo establecido por
      el art. 6º de sus Estatutos, comprometiéndose a su cumplimiento.
    \end{enumerate}
  \item
    \begin{enumerate}
    \def\labelenumi{\alph{enumi})}
    \setcounter{enumi}{4}
    \tightlist
    \item
      Convocar a elecciones cada tres (3) años para la renovación total
      de la Comisión Directiva, Comisión Revisora de Cuentas y Junta
      Representativa.
    \end{enumerate}
  \item
    \begin{enumerate}
    \def\labelenumi{\alph{enumi})}
    \setcounter{enumi}{5}
    \tightlist
    \item
      Disponer la convocatoria a Asamblea extraordinaria cuando por la
      naturaleza del y/o los asuntos a considerar, así correspondiere
      estatutariamente, de acuerdo al \protect\hyperlink{art40}{artículo
      40º}, del Capítulo VI, del presente Estatuto.
    \end{enumerate}
  \item
    \begin{enumerate}
    \def\labelenumi{\alph{enumi})}
    \setcounter{enumi}{6}
    \tightlist
    \item
      Aceptar o rechazar la renuncia de sus miembros e incorporar a los
      suplentes.
    \end{enumerate}
  \item
    \begin{enumerate}
    \def\labelenumi{\alph{enumi})}
    \setcounter{enumi}{7}
    \tightlist
    \item
      Comprar y/o vender bienes, inmuebles o semovientes, gravarlos o
      tomarlos en arrendamiento; celebrar todo tipo de contratos o
      convenciones lícitas; obtener y cancelar empréstitos, realizar
      transacciones, solicitar, aceptar o rechazar donaciones, subsidios
      o legados de particulares, de instituciones o del Estado. Para
      comprar, vender o gravar bienes inmuebles, rechazar legados,
      donaciones o subsidios, la Comisión Directiva deberá estar
      autorizada por una Asamblea General Extraordinaria.
    \end{enumerate}
  \item
    \begin{enumerate}
    \def\labelenumi{\roman{enumi})}
    \tightlist
    \item
      Proyectar el reemplazo, modificación o corrección de este
      Estatuto, la disolución de la Entidad o su fusión con otra u otras
      que persigan similares propósitos. La aprobación de estos
      proyectos y consiguiente convocatoria a Asamblea General
      Extraordinaria, requerirá el voto favorable de los dos tercios
      (2/3) de los miembros presentes en la respectiva sesión de la
      Comisión Directiva.
    \end{enumerate}
  \item
    \begin{enumerate}
    \def\labelenumi{\alph{enumi})}
    \setcounter{enumi}{9}
    \tightlist
    \item
      Establecer todas aquellas secciones accesorias que correspondan a
      los fines de la Institución, nombrando Delegados, profesores,
      Técnicos, Encargados, Capitanes y llenando todo cargo que se
      considere necesario, sea éste rentado u honorario.
    \end{enumerate}
  \item
    \begin{enumerate}
    \def\labelenumi{\alph{enumi})}
    \setcounter{enumi}{10}
    \tightlist
    \item
      Organizar concursos, discernir distinciones, premios y diplomas,
      patrocinar publicaciones que sirvan a la información y elevación
      cultural de los socios y que beneficien al Club.
    \end{enumerate}
  \item
    \begin{enumerate}
    \def\labelenumi{\alph{enumi})}
    \setcounter{enumi}{11}
    \tightlist
    \item
      Nombrar al personal administrativo, fijando su sueldo y sus
      funciones; decretar suspensiones, cesantías y exoneraciones del
      mismo.
    \end{enumerate}
  \item
    \begin{enumerate}
    \def\labelenumi{\roman{enumi})}
    \setcounter{enumi}{99}
    \tightlist
    \item
      Suscribir convenios con otras instituciones o particulares, por la
      transferencia de deportistas aficionados o rentados, como así
      contratos con éstos, con técnicos y profesionales; todo ello, en
      arreglo a las leyes vigentes en oportunidad de producirse tales
      hechos y sin comprometer, gravemente, el patrimonio de la
      Institución.
    \end{enumerate}
  \item
    \begin{enumerate}
    \def\labelenumi{\alph{enumi})}
    \setcounter{enumi}{12}
    \tightlist
    \item
      Fijar y modificar las cuotas sociales, de ingreso, de reingreso, y
      tasas diversas para cada categoría.
    \end{enumerate}
  \item
    \begin{enumerate}
    \def\labelenumi{\alph{enumi})}
    \setcounter{enumi}{13}
    \tightlist
    \item
      Fijar y modificar el valor de credenciales, entradas, alquileres y
      otros servicios sociales que el Club preste, efectuando su
      cobranza en la forma más conveniente.
    \end{enumerate}
  \item
    \begin{enumerate}
    \def\labelenumi{\alph{enumi})}
    \setcounter{enumi}{14}
    \tightlist
    \item
      Decretar suscripciones de socios y sus condiciones dentro de las
      normas del presente Estatuto.
    \end{enumerate}
  \item
    \begin{enumerate}
    \def\labelenumi{\alph{enumi})}
    \setcounter{enumi}{15}
    \tightlist
    \item
      Disponer la explotación directa o indirecta, ésta a través de
      concesiones, de comedores, restaurantes, buffets, quioscos, etc.,
      dentro de las instalaciones del Club, suscribiendo los respectivos
      contratos y llamando a licitación o concursos, cuando la
      importancia del negocio lo requiera.
    \end{enumerate}
  \item
    \begin{enumerate}
    \def\labelenumi{\alph{enumi})}
    \setcounter{enumi}{16}
    \tightlist
    \item
      Contratar publicidad para el Club o para terceros, de por sí o por
      medio y a través de concesiones, dentro de las instalaciones de
      aquel o en órganos o agencias en que resultare conveniente
      hacerlo, suscribiendo para ello contratos pertinentes.
    \end{enumerate}
  \item
    \begin{enumerate}
    \def\labelenumi{\alph{enumi})}
    \setcounter{enumi}{17}
    \tightlist
    \item
      Aceptar o rechazar la renuncia de socios, como así admitir,
      rechazar, decretar cesantías, expulsiones o limitaciones en su
      número.
    \end{enumerate}
  \item
    \begin{enumerate}
    \def\labelenumi{\alph{enumi})}
    \setcounter{enumi}{18}
    \tightlist
    \item
      Sancionar a socios, personal administrativo, técnico o
      profesional, como así también a deportistas, profesionales o
      aficionados, que los representen, todo conforme a este Estatuto y
      reglamentaciones vigentes en el Club y en Asociaciones o
      Federaciones a las que la Institución esté afiliada y, por lo
      tanto, comprometida a respetar, como así también a las leyes del
      Estado vigentes.
    \end{enumerate}
  \item
    \begin{enumerate}
    \def\labelenumi{\alph{enumi})}
    \setcounter{enumi}{19}
    \tightlist
    \item
      Designar los miembros de Departamentos y Subcomisiones, aceptar o
      rechazar sus renuncias y disponer sus cesantías, como así también
      asignar las partidas necesarias para su normal funcionamiento.
    \end{enumerate}
  \item
    \begin{enumerate}
    \def\labelenumi{\alph{enumi})}
    \setcounter{enumi}{20}
    \tightlist
    \item
      Conceder licencias a sus propios miembros, socios, empleados,
      etc., como así negarlas cuando no procedieren.
    \end{enumerate}
  \item
    \begin{enumerate}
    \def\labelenumi{\alph{enumi})}
    \setcounter{enumi}{21}
    \tightlist
    \item
      Dictar reglamentaciones, órdenes y resoluciones que hagan al mejor
      desenvolvimiento de la Institución, realizando todos los actos
      dirigidos al cumplimiento de sus fines esenciales, siempre que no
      estén prohibidos por este Estatuto y/o por leyes del Estado.
    \end{enumerate}
  \item
    \begin{enumerate}
    \def\labelenumi{\alph{enumi})}
    \setcounter{enumi}{22}
    \tightlist
    \item
      Invertir en obras de utilidad deportiva o cultural el remanente
      líquido que obtenga del fútbol (art. 6º Estatuto AFA).
    \end{enumerate}
  \item
    \begin{enumerate}
    \def\labelenumi{\alph{enumi})}
    \setcounter{enumi}{23}
    \tightlist
    \item
      Someter toda diferencia o litigio que pudiera surgir con la
      Asociación del Fútbol Argentino, otras Asociaciones o clubes
      afiliados a éstas, ante un Tribunal Arbitral, nombrado de común
      acuerdo (Arts. 59 Estatuto FIFA y 6º Estatuto AFA), con antelación
      a su planteo ante los Tribunales de Justicia.
    \end{enumerate}
  \item
    \begin{enumerate}
    \def\labelenumi{\alph{enumi})}
    \setcounter{enumi}{24}
    \tightlist
    \item
      Dar a publicidad, en forma semestral, a través de los órganos de
      difusión oficiales del Club o por los medios que estime oportunos,
      el estado económico--financiero de la Institución.
    \end{enumerate}
  \end{itemize}
\end{itemize}

\begin{itemize}
\tightlist
\item
  \textbf{Art. 69º} Queda absolutamente prohibido a los miembros de la
  Comisión Directiva bajo pena de suspensión, dar explicaciones o
  noticias de cualquier género sobre resoluciones, discusiones,
  votaciones o deliberaciones habidas en el seno de la misma, o
  comentarios con personas ajenas a la Comisión Directiva, siempre que
  ello se derive un perjuicio para la institución.
\end{itemize}

\chapter{DEL PRESIDENTE}\label{del-presidente}

\begin{itemize}
\tightlist
\item
  \textbf{Art. 70º} El Presidente representa a la institución en todos
  sus actos, ya sean estos legales, deportivos o sociales y ejerce
  además todas las otras funciones inherentes al cargo y representación
  que inviste, teniendo las siguientes atribuciones y obligaciones:

  \begin{itemize}
  \item
    \begin{enumerate}
    \def\labelenumi{\alph{enumi})}
    \tightlist
    \item
      Presidir las Asambleas y reuniones de la Comisión Directiva, como
      así también, cuando lo creyere conveniente, las sesiones de las
      Subcomisiones y delegaciones del Club, firmando sus actas.
    \end{enumerate}
  \item
    \begin{enumerate}
    \def\labelenumi{\alph{enumi})}
    \setcounter{enumi}{1}
    \tightlist
    \item
      Dirigir y organizar las discusiones y votaciones en las asambleas
      y Comisión Directiva y Subcomisiones que presidiere, resolviendo
      en todo asunto en caso de empate, con su voto.
    \end{enumerate}
  \item
    \begin{enumerate}
    \def\labelenumi{\alph{enumi})}
    \setcounter{enumi}{2}
    \tightlist
    \item
      Dirigir a la administración de la Institución, con facultades para
      intervenir en todos los detalles de ésta.
    \end{enumerate}
  \item
    \begin{enumerate}
    \def\labelenumi{\alph{enumi})}
    \setcounter{enumi}{3}
    \tightlist
    \item
      Terciar en las discusiones de las asambleas, Comisión Directiva y
      Subcomisiones, con voz y voto, dejando la Presidencia en estos
      casos a quien corresponda.
    \end{enumerate}
  \item
    \begin{enumerate}
    \def\labelenumi{\alph{enumi})}
    \setcounter{enumi}{4}
    \tightlist
    \item
      Ejecutar y hacer ejecutar las resoluciones que adaptaren las
      Asambleas o Comisión Directiva.
    \end{enumerate}
  \item
    \begin{enumerate}
    \def\labelenumi{\alph{enumi})}
    \setcounter{enumi}{5}
    \tightlist
    \item
      Firmar los documentos y correspondencia de carácter oficial del
      Club, Balance y Memoria anuales, como así también las órdenes de
      pago, chequeras, giros y obligaciones privadas o bancarias,
      debidamente autorizadas.
    \end{enumerate}
  \item
    \begin{enumerate}
    \def\labelenumi{\alph{enumi})}
    \setcounter{enumi}{6}
    \tightlist
    \item
      Otorgar, aceptar y firmar los instrumentos públicos y documentos
      privados y relacionados con los actos o contratos autorizados por
      la Comisión Directiva o Asambleas.
    \end{enumerate}
  \item
    \begin{enumerate}
    \def\labelenumi{\alph{enumi})}
    \setcounter{enumi}{7}
    \tightlist
    \item
      Convocar a la Comisión Directiva cuando lo estime conveniente o a
      pedido de seis (6) miembros de la misma, debiendo en todos los
      casos indicar el o los asuntos a tratarse. La convocatoria
      realizada por Secretaría General será dentro de las 48 horas de
      formuladas las peticiones.
    \end{enumerate}
  \item
    \begin{enumerate}
    \def\labelenumi{\roman{enumi})}
    \tightlist
    \item
      Convocar a Asamblea cuando lo disponga la Comisión Directiva.
    \end{enumerate}
  \item
    \begin{enumerate}
    \def\labelenumi{\alph{enumi})}
    \setcounter{enumi}{9}
    \tightlist
    \item
      Resolver por sí solo en cualquier asunto o caso especial que por
      su urgencia no pudiera ser tratado por la Comisión Directiva, con
      cargo a rendir cuenta en la primera sesión que ésta celebre.
    \end{enumerate}
  \item
    \begin{enumerate}
    \def\labelenumi{\alph{enumi})}
    \setcounter{enumi}{10}
    \tightlist
    \item
      Redactar juntamente con el Secretario General, la Memoria anual
      que debe presentarse a la Asamblea General Ordinaria.
    \end{enumerate}
  \item
    \begin{enumerate}
    \def\labelenumi{\alph{enumi})}
    \setcounter{enumi}{11}
    \tightlist
    \item
      Intervenir y fiscalizar en cualquier momento las actividades de
      diverso orden de la sección administrativa y Subcomisiones.
    \end{enumerate}
  \item
    \begin{enumerate}
    \def\labelenumi{\alph{enumi})}
    \setcounter{enumi}{12}
    \tightlist
    \item
      El Presidente o los Vicepresidentes, en su caso, serán
      solidariamente responsables con el Tesorero de los pagos que
      efectúen y con el Secretario General, de los actos que suscriban.
    \end{enumerate}
  \end{itemize}
\end{itemize}

\chapter{DEL VICEPRESIDENTE 1º}\label{del-vicepresidente-1}

\begin{itemize}
\item
  \textbf{Art. 71º} El Vicepresidente 1º auxiliará y colaborará con e l
  Presidente en sus funciones. En virtud de su cargo, el Vicepresidente
  1º supervisará el funcionamiento del fútbol de la Institución, tanto
  profesional como amateur, y las actividades del predio ``La
  Agustina''.
\item
  \textbf{Art. 72º} El Vicepresidente 1º reemplazará al Presidente en
  caso de fallecimiento, ausencia, enfermedad o renuncia, con las mismas
  funciones, atribuciones y obligaciones conferidas y exigidas a éste y
  tendrá voz y voto en las Asambleas y reuniones de la Comisión
  Directiva, siempre que no se encuentre en ejercicio de la Presidencia.
\end{itemize}

\chapter{DEL VICEPRESIDENTE 2º}\label{del-vicepresidente-2}

\begin{itemize}
\tightlist
\item
  \textbf{Art. 73º} El Vicepresidente 2º auxiliará y colaborará con e l
  Presidente y Vicepresidente 1º en el desempeño de sus funciones. En
  virtud de su cargo, el Vicepresidente 2º supervisará el funcionamiento
  de las restantes disciplinas que se desarrollen en la Institución, las
  actividades sociales y presidirá la Junta Representativa.
\end{itemize}

\begin{itemize}
\tightlist
\item
  \textbf{Art. 74º} El Vicepresidente 2º reemplazará al Presidente en
  caso de fallecimiento, ausencia, enfermedad o renuncia del Presidente
  y Vicepresidente 1º, con las mismas funciones, atribuciones y
  obligaciones conferidas al Presidente tendrá voz y voto en las
  Asambleas y reuniones de Comisión Directiva, siempre que no se
  encuentre en ejercicio de la Presidencia.
\end{itemize}

\begin{itemize}
\tightlist
\item
  \textbf{Art. 75º} En caso de renuncia, muerte, ausencia o enfermedad
  del Presidente, Vicepresidente 1º y Vicepresidente 2º, se hará cargo
  de la Presidencia, el Vocal de la Comisión Directiva que la misma
  designe, quien deberá dentro del término de veinte (20) días de
  haberse hecho cargo llamar a Asamblea Extraordinaria para la elección
  de Presidente y Vicepresidentes, siempre que no faltase menos de
  sesenta (60) días corridos para la culminación del período.
\end{itemize}

\chapter{DEL SECRETARIO GENERAL}\label{del-secretario-general}

\begin{itemize}
\tightlist
\item
  \textbf{Art. 76º} El Secretario General es el colaborador inmediato
  del Presidente. Sus obligaciones, deberes y atribuciones son:

  \begin{itemize}
  \item
    \begin{enumerate}
    \def\labelenumi{\alph{enumi})}
    \tightlist
    \item
      Representar al Presidente y Vicepresidente dentro de los locales
      de la Institución, cuando ellos no estuvieren presentes, así como
      ejercer la superintendencia inmediata de todo el personal de la
      Institución.
    \end{enumerate}
  \item
    \begin{enumerate}
    \def\labelenumi{\alph{enumi})}
    \setcounter{enumi}{1}
    \tightlist
    \item
      Refrendar la firma del Presidente en todos los documentos
      sociales, con excepción de cheques y libranzas.
    \end{enumerate}
  \item
    \begin{enumerate}
    \def\labelenumi{\alph{enumi})}
    \setcounter{enumi}{2}
    \tightlist
    \item
      Entender en todos los asuntos, estudiarlos, tratarlos, ordenarlos,
      someterlos al Presidente y por su intermedio a la Comisión
      Directiva, excepto los asuntos de Tesorería.
    \end{enumerate}
  \item
    \begin{enumerate}
    \def\labelenumi{\alph{enumi})}
    \setcounter{enumi}{3}
    \tightlist
    \item
      Asistir a las Asambleas y sesiones de la Comisión Directiva.
    \end{enumerate}
  \item
    \begin{enumerate}
    \def\labelenumi{\alph{enumi})}
    \setcounter{enumi}{4}
    \tightlist
    \item
      Recibir las solicitudes de socios, dándoles el curso
      correspondiente y someterlas a la consideración de la Comisión
      Directiva.
    \end{enumerate}
  \item
    \begin{enumerate}
    \def\labelenumi{\alph{enumi})}
    \setcounter{enumi}{5}
    \tightlist
    \item
      Recibir y darle curso a toda la correspondencia que llegue a la
      Institución y redactar o hacer redactar las respuestas y toda la
      correspondencia a dirigir, conservando copias.
    \end{enumerate}
  \item
    \begin{enumerate}
    \def\labelenumi{\alph{enumi})}
    \setcounter{enumi}{6}
    \tightlist
    \item
      Cursar la convocatoria a reunión o sesión de la Comisión Directiva
      para los días que ésta hubiese fijado ordinariamente o cuando el
      Presidente lo dispusiera de acuerdo al
      \protect\hyperlink{art67}{artículo 67º} inciso h.
    \end{enumerate}
  \item
    \begin{enumerate}
    \def\labelenumi{\alph{enumi})}
    \setcounter{enumi}{7}
    \tightlist
    \item
      Expedir, con intervención de la Tesorería, los carnets o
      credenciales suscribiéndolas con su firma.
    \end{enumerate}
  \item
    \begin{enumerate}
    \def\labelenumi{\roman{enumi})}
    \tightlist
    \item
      Supervisar los registros de socios, formar o hacer formar con
      aquellos, el fichero de socios en actividad, de ex socios y
      establecer el padrón de asociados.
    \end{enumerate}
  \item
    \begin{enumerate}
    \def\labelenumi{\alph{enumi})}
    \setcounter{enumi}{9}
    \tightlist
    \item
      Tomar a su cargo las actas si por cualquier causa no estuviere el
      Secretario de Actas e Información Institucional, ni el sustituto,
      pudiendo para ello, solicitar la colaboración de cualquier miembro
      de la Comisión Directiva o bien un empleado del Club.
    \end{enumerate}
  \item
    \begin{enumerate}
    \def\labelenumi{\alph{enumi})}
    \setcounter{enumi}{10}
    \tightlist
    \item
      Recopilar los documentos o informes necesarios para la redacción
      de la Memoria anual.
    \end{enumerate}
  \item
    \begin{enumerate}
    \def\labelenumi{\alph{enumi})}
    \setcounter{enumi}{11}
    \tightlist
    \item
      Conservar los trofeos del Club, de los que llevará un registro
      especial de todos, de cuyos efectos será responsable.
    \end{enumerate}
  \end{itemize}
\end{itemize}

\chapter{DEL PROSECRETARIO GENERAL}\label{del-prosecretario-general}

\begin{itemize}
\item
  \textbf{Art. 77º} La función específica del Prosecretario será la de
  colaborar con el Secretario General cuantas veces y en cuantos actos
  fuere necesario, como así reemplazarlo en caso de fallecimiento,
  enfermedad, ausencia, renuncia u otro impedimento.
\item
  \textbf{Art. 78º} En caso de ausencia transitoria del Secretario y
  Pro-Secretario, la Comisión Directiva designará un reemplazante de los
  mismos, ``ad hoc''. Si aquella es definitiva, la Comisión Directiva
  elegirá entre los vocales, sus reemplazantes.
\end{itemize}

\chapter{DEL TESORERO}\label{del-tesorero}

\begin{itemize}
\item
  \textbf{Art. 79º} El Tesorero es el responsable de los fondos y
  valores de la institución. Cuidará de la integridad de los ingresos y
  de la procedencia y corrección de los egresos, con arreglo a las
  disposiciones de este Estatuto, del Reglamento Interno y de las
  resoluciones de Comisión Directiva que se dicten en su consecuencia.
\item
  \textbf{Art. 80º} El Tesorero es autoridad competente para intervenir
  en todo acto u operación relativa a bienes, fondos, valores y demás
  cosas del patrimonio social. Está facultado para tomar cualquier
  medida en defensa de los intereses sociales, con cargo de rendir
  cuenta a Comisión Directiva.
\item
  \textbf{Art. 81º} Son deberes y atribuciones del Tesorero:

  \begin{itemize}
  \item
    \begin{enumerate}
    \def\labelenumi{\alph{enumi})}
    \tightlist
    \item
      La recaudación y custodia de todos los fondos de la Institución,
      llevando los libros de Caja, Diario, Mayor e Inventario y los
      auxiliares necesarios.
    \end{enumerate}
  \item
    \begin{enumerate}
    \def\labelenumi{\alph{enumi})}
    \setcounter{enumi}{1}
    \tightlist
    \item
      Verificar los pagos correspondientes a la administración, siempre
      que ellos hayan sido autorizados por la Comisión Directiva.
    \end{enumerate}
  \item
    \begin{enumerate}
    \def\labelenumi{\alph{enumi})}
    \setcounter{enumi}{2}
    \tightlist
    \item
      Depositar en el y/o los bancos que indique la Comisión Directiva,
      en cuenta corriente o en otra forma, a nombre del Club y a la
      orden conjunta del Presidente, Secretario y Tesorero todos los
      fondos sociales, a excepción de aquellos que fueren indispensables
      para atender las necesidades diarias y menores del Club.
    \end{enumerate}
  \item
    \begin{enumerate}
    \def\labelenumi{\alph{enumi})}
    \setcounter{enumi}{3}
    \tightlist
    \item
      Juntamente con la Comisión Directiva, la iniciativa en materia
      económica y financiera, elaboración del presupuesto de inversiones
      y gastos y el cálculo de recurso de cada ejercicio.
    \end{enumerate}
  \item
    \begin{enumerate}
    \def\labelenumi{\alph{enumi})}
    \setcounter{enumi}{4}
    \tightlist
    \item
      Presentar mensualmente a la Comisión Directiva una balance de
      caja, acompañando los respectivos comprobantes, y anualmente un
      balance general y otro de caja al 30 de Abril, dividido en tantas
      partes como sea necesario. Estos balances deberán ser presentados
      a la Comisión Directiva dentro de los 30 días de cerrado el
      ejercicio.
    \end{enumerate}
  \item
    \begin{enumerate}
    \def\labelenumi{\alph{enumi})}
    \setcounter{enumi}{5}
    \tightlist
    \item
      Firmar conjuntamente con el Presidente, los cheques, libranzas u
      otros documentos de crédito y por sí los recibos que deba otorgar
      el club a los socios o particulares.
    \end{enumerate}
  \item
    \begin{enumerate}
    \def\labelenumi{\alph{enumi})}
    \setcounter{enumi}{6}
    \tightlist
    \item
      Recibir del Secretario General las solicitudes de socios, informar
      en caso de tratarse de socios morosos, a fin de aplicar el
      \protect\hyperlink{art20}{artículo 20º}.
    \end{enumerate}
  \item
    \begin{enumerate}
    \def\labelenumi{\alph{enumi})}
    \setcounter{enumi}{7}
    \tightlist
    \item
      Previa autorización de la Comisión Directiva, verificará los pagos
      correspondientes a la indemnización social u elevará todos los
      meses la nómina de los socios que se encuentren en estado de
      cesación de pago.
    \end{enumerate}
  \item
    \begin{enumerate}
    \def\labelenumi{\roman{enumi})}
    \tightlist
    \item
      Dará cuenta inmediata y por escrito al Presidente, de cualquier
      irregularidad que observare en los libros o en el manejo de los
      fondos de la Institución.
    \end{enumerate}
  \item
    \begin{enumerate}
    \def\labelenumi{\alph{enumi})}
    \setcounter{enumi}{9}
    \tightlist
    \item
      Llevar un registro de pago de la cuota de socios, el que debe
      estar al día y a disposición de la Comisión Directiva y la
      Comisión Revisora de Cuentas.
    \end{enumerate}
  \item
    \begin{enumerate}
    \def\labelenumi{\alph{enumi})}
    \setcounter{enumi}{10}
    \tightlist
    \item
      Realizar todos los demás actos y gestiones propias de la función,
      siempre que no le estén expresa o implícitamente prohibidos por el
      Estatuto y/o las reglamentaciones dictadas en su consecuencia.
    \end{enumerate}
  \end{itemize}
\end{itemize}

\chapter{DEL PROTESORERO}\label{del-protesorero}

\begin{itemize}
\item
  \textbf{Art. 82º} La función específica del Protesorero será la de
  colaborar con el Tesorero cuantas veces y en cuantos actos fuere
  necesario, como así reemplazarlo en caso de fallecimiento, enfermedad,
  ausencia, renuncia u otro impedimento.
\item
  \textbf{Art. 83º} En caso de fallecimiento, enfermedad o renuncia del
  Tesorero y Protesorero, la Comisión Directiva elegirá de entre los
  vocales al que deba ejercer la Tesorería y que reunirá las condiciones
  de idoneidad y antigüedad de acuerdo al
  \protect\hyperlink{art58}{artículo 58º}.
\end{itemize}

\chapter{DEL SECRETARIO DE ACTAS E INFORMACIÓN
INSTITUCIONAL}\label{del-secretario-de-actas-e-informacion-institucional}

\begin{itemize}
\item
  \textbf{Art. 84º} El Secretario de Actas e Información Institucional
  será la persona que se encargará de todos los trámites atinentes a la
  presentación a las reuniones de la Comisión Directiva o Asamblea de
  Asociados, de las actas que se levantaron y que estarán a su exclusivo
  cargo. Confeccionará el borrador de las mismas que hará transcribir
  por empleado de la Institución y verificará luego su pase al libro.
  Será encargado de su lectura ante quien corresponda y una vez
  aprobadas, las hará suscribir por el Presidente y el Secretario Ge
  neral si se tratara de Actas de Reuniones de la Comisión Directiva y
  por dos asociados designados al efecto, si se tratase de Asambleas de
  asociados. Firmadas las mismas será el responsable a la guarda y
  control de los libros respectivos.
\item
  \textbf{Art. 85º} Será asimismo, el Vocero oficial de la Institución,
  dando a conocer por medio de boletines, la información de más
  importancia y trascendencia que tenga relación con la vida y marcha
  del Club. Para el desempeño de tales tareas, podrá designar los
  colaboradores que estime convenientes y necesarios. Estará igualmente
  facultado a desautorizar informaciones que no provengan del
  Presidente, Vice-Presidente y/o Secretario General y que tiendan a
  deformar la realidad o que conspiren contra los objetivos
  fundamentales de la Institución.
\end{itemize}

\chapter{DEL PROSECRETARIO DE ACTAS E INFORMACIÓN
INSTITUCIONAL}\label{del-prosecretario-de-actas-e-informacion-institucional}

\begin{itemize}
\item
  \textbf{Art. 86º} La función específica del Prosecretario de Actas e
  Información Institucional será la de colaborar con el Secretario de
  Actas e Información Institucional cuantas veces y en cuantos actos
  fuere necesario, como así reemplazarlo en caso de fallecimiento,
  enfermedad, ausencia, renuncia u otro impedimento.
\item
  \textbf{Art. 87º} En caso de ausencia transitoria del Secretario de
  Actas e Información Institucional y del Pro-Secretario de Actas e
  Información Institucional, el Secretario General, con acuerdo de la
  Comisión Directiva, designará un reemplazante de los mismos, ``ad
  hoc''. Si aquella es definitiva, el Secretario General elegirá entre
  los vocales, sus reemplazantes.
\end{itemize}

\chapter{DEL SECRETARIO DE RECURSOS
FINANCIEROS}\label{del-secretario-de-recursos-financieros}

\begin{itemize}
\item
  \textbf{Art. 88º} El Secretario de Recursos Financieros será el
  responsable de obtener los recursos extraordinarios (sponsors,
  donaciones, publicidad, eventos, préstamos, etc) necesarios para que
  completando a los recursos ordinarios (cuotas sociales, entradas de
  los espectáculos deportivos, transferencia de jugadores) permitan
  llevar adelante el Presupuesto anual aprobado por la Comisión
  Directiva.
\item
  \textbf{Art. 89º} Será asimismo, el responsable de las relaciones del
  Club ante organismos oficiales y entidades privadas que mantengan
  contacto con la Institución, coordinando con las distintas
  Subcomisiones, todos los aspectos inherentes a la atención de
  delegaciones visitantes.
\end{itemize}

\chapter{DEL PROSECRETARIO DE RECURSOS
FINANCIEROS}\label{del-prosecretario-de-recursos-financieros}

\begin{itemize}
\item
  \textbf{Art. 90º} La función específica del Prosecretario de Recursos
  Financieros será la de colaborar con el Secretario de Recursos
  Financieros cuantas veces y en cuantos actos fuere necesario, como así
  reemplazarlo en caso de fallecimiento, enfermedad, ausencia, renuncia
  u otro impedimento.
\item
  \textbf{Art. 91º} En caso de fallecimiento, enfermedad o renuncia del
  Secretario de Recursos Financieros y Humanos y Prosecretario de
  Recursos Financieros y Humanos, la Comisión Directiva elegirá de entre
  los vocales al que deba ejercer la Tesorería y que reunirá las
  condiciones de idoneidad y antigüedad de acuerdo al
  \protect\hyperlink{art55}{artículo 55º}.
\end{itemize}

\chapter{DE LOS VOCALES}\label{de-los-vocales}

\begin{itemize}
\tightlist
\item
  \textbf{Art. 92º} Los Vocales son los miembros de la Comisión
  Directiva que ésta necesita para sus deliberaciones y consejo,
  teniendo durante el desarrollo de las mismas, voz y voto. Son deberes
  y atribuciones de los Vocales:

  \begin{itemize}
  \item
    \begin{enumerate}
    \def\labelenumi{\alph{enumi})}
    \tightlist
    \item
      Concurrir con puntualidad a las sesiones o reuniones de la
      Comisión Directiva.
    \end{enumerate}
  \item
    \begin{enumerate}
    \def\labelenumi{\alph{enumi})}
    \setcounter{enumi}{1}
    \tightlist
    \item
      Desempeñar todas las funciones o comisiones que le encomiende la
      Comisión Directiva.
    \end{enumerate}
  \item
    \begin{enumerate}
    \def\labelenumi{\alph{enumi})}
    \setcounter{enumi}{2}
    \tightlist
    \item
      Participar activamente en la vida de la Institución, interesándose
      en todos sus asuntos.
    \end{enumerate}
  \item
    \begin{enumerate}
    \def\labelenumi{\alph{enumi})}
    \setcounter{enumi}{3}
    \tightlist
    \item
      Solicitar al Presidente que convoque a sesión de Comisión
      Directiva de acuerdo al \protect\hyperlink{art68}{artículo 68º}
      inciso h.
    \end{enumerate}
  \item
    \begin{enumerate}
    \def\labelenumi{\alph{enumi})}
    \setcounter{enumi}{4}
    \tightlist
    \item
      Hacer guardar el orden en las dependencias de la Institución en
      colaboración con las demás autoridades o por sí solo en ausencia
      de aquellas y tomar las medidas que fueren necesarias o de
      carácter indispensable para salvar dificultades apremiantes con
      cargo de rendir cuentas a la Comisión Directiva en la primera
      sesión.
    \end{enumerate}
  \item
    \begin{enumerate}
    \def\labelenumi{\alph{enumi})}
    \setcounter{enumi}{5}
    \tightlist
    \item
      Ejercer la administración de la Subcomisión a su cargo si fuera
      designado por la Comisión Directiva, como asimismo, desarrollar
      las tareas necesarias para la difusión de las actividades que se
      desarrollan dentro de las esferas de dicha Subcomisión.
    \end{enumerate}
  \item
    \begin{enumerate}
    \def\labelenumi{\alph{enumi})}
    \setcounter{enumi}{6}
    \tightlist
    \item
      Guardar absoluta reserva sobre cualquier asunto tratado durante
      las sesiones de la Comisión Directiva o Subcomisiones que
      integraren.
    \end{enumerate}
  \item
    \begin{enumerate}
    \def\labelenumi{\alph{enumi})}
    \setcounter{enumi}{7}
    \tightlist
    \item
      Los Vocales tienen la obligación de cumplir con las funciones
      específicas que les asigne la Comisión Directiva, acorde a lo
      establecido por el \protect\hyperlink{art59}{artículo 59º}; en
      caso de no hacerlo, serán pasibles de aplicarles las sanciones que
      establecen los \protect\hyperlink{art66}{artículos 66º} y
      \protect\hyperlink{art69}{69º} de estos Estatutos, no solo en lo
      referente a la asistencia, sino también a las faltas cometidas en
      el desempeño del cargo encomendado.
    \end{enumerate}
  \item
    \begin{enumerate}
    \def\labelenumi{\roman{enumi})}
    \tightlist
    \item
      Sin perjuicio de lo mencionado en el inciso anterior, la Comisión
      Directiva designará, de manera idónea, un vocal que deberá
      participar, por sí o por el representante legal que se designe, en
      todos los asuntos judiciales en que la Institución sea parte e
      informar mensualmente a la Comisión Directiva sobre el estado de
      los mismos.
    \end{enumerate}
  \end{itemize}
\end{itemize}

\begin{itemize}
\tightlist
\item
  \textbf{Art. 93º} Las vacantes que se produzcan en los cargos de
  Vocales por las causas previstas en los artículos precedentes, como
  así también en caso de renuncia, enfermedad o fallecimiento, se
  llenará con miembros de la Junta Representativa, respetándose el rango
  por el cual fueron electos en los comicios de asociados.
\end{itemize}

\chapter{DE LAS SUBCOMISIONES}\label{de-las-subcomisiones}

\begin{itemize}
\item
  \textbf{Art. 94º} La Comisión Directiva dispondrá la constitución de
  las Subcomisiones que estime conveniente para dirigir las distintas
  actividades y deportes dentro del Club.
\item
  \textbf{Art. 95º} Las Subcomisiones serán integradas exclusivamente
  por socios, sin distinción de sexo, mayores de dieciocho (18) años de
  edad y en las condiciones previstas por el presente Estatuto.
\item
  \textbf{Art. 96º} Las Subcomisiones serán dirigidas y administradas
  por un Presidente, un Secretario y un Tesorero, nombrados por la
  Comisión Directiva a propuesta de los asociados.
\item
  \textbf{Art. 97º} Cuando un grupo de asociados desee agregar una nueva
  disciplina deportiva o actividad social o cultural a las ya existentes
  en la entidad, deberá solicitar por nota dirigida al Presidente y
  Comisión Directiva, la autorización correspondiente. La mencionada
  nota deberá estar firmada, como mínimo, por el cinco (5) por ciento
  del caudal societario del Club, e incluir en la misma, una terna de
  asociados, de la que la Comisión Directiva elegirá el Presidente, el
  Secretario y el Tesorero que administrará el nuevo Departamento o
  Subcomisión.
\item
  \textbf{Art. 98º} Las Subcomisiones se reunirán por lo menos cada
  quince (15) días. Su quórum legal será la mitad más uno de sus
  miembros y las decisiones se tomarán por simple mayoría de votos.
  Llevarán un libro de actas.
\end{itemize}

\begin{itemize}
\item
  \textbf{Art. 99º} Las Subcomisiones, se deben ayuda recíproca. Todo
  entredicho que se suscitare entre ellas, será resuelto, sin apelación
  por la Junta Representativa.
\item
  \textbf{Art. 100º} Las Subcomisiones pueden, con asentimiento de la
  Comisión Directiva, establecer un aporte monetario con que cada
  afiliado debe contribuir para solventar los gastos de mantenimiento.
  Los fondos que en tal concepto se recauden o que por cualquier título
  ingresen a la Subcomisión, serán administrados por ella, debiendo dar
  cuenta de su ingreso o inversión a la Comisión Directiva mensualmente.
\item
  \textbf{Art. 101º} Las atribuciones y obligaciones de las
  Subcomisiones son:

  \begin{itemize}
  \item
    \begin{enumerate}
    \def\labelenumi{\alph{enumi})}
    \tightlist
    \item
      Reglamentar la práctica de las actividades sociales y deportivas
      respectivas.
    \end{enumerate}
  \item
    \begin{enumerate}
    \def\labelenumi{\alph{enumi})}
    \setcounter{enumi}{1}
    \tightlist
    \item
      Recaudar los fondos necesarios para la práctica de los mismos.
    \end{enumerate}
  \item
    \begin{enumerate}
    \def\labelenumi{\alph{enumi})}
    \setcounter{enumi}{2}
    \tightlist
    \item
      Concertar justas, torneos, partidos, efectuar reuniones de
      carácter social, cultural, deportivo o benéfico, con previa
      autorización de la Comisión Directiva.
    \end{enumerate}
  \item
    \begin{enumerate}
    \def\labelenumi{\alph{enumi})}
    \setcounter{enumi}{3}
    \tightlist
    \item
      Administrar los fondos recaudados por esos medios.
    \end{enumerate}
  \end{itemize}
\item
  \textbf{Art. 102º} Las Subcomisiones no pueden firmar contratos ni
  obligar a la Institución. Tampoco pueden nombrar personal a sueldo,
  debiendo proponerlos a la Comisión Directiva quien otorgará el
  correspondiente asentimiento y quien a su vez puede declararlo
  cesante.
\item
  \textbf{Art. 103º} Al finalizar cada período económico, las
  Subcomisiones y Departamentos elevarán un balance y memoria detallando
  las actividades cumplidas durante dicho período.
\end{itemize}

\chapter{DE LA JUNTA REPRESENTATIVA}\label{de-la-junta-representativa}

\begin{itemize}
\tightlist
\item
  \textbf{Art. 104º} La Junta Representativa actuará como mediador en
  las cuestiones que se susciten entre los socios, o entre estos y la
  Comisión Directiva, velando por la defensa y protección de los
  derechos de los asociados. Asimismo, arbitrará en los conflictos que
  se susciten entre las Subcomisiones, acorde a lo establecido por el
  \protect\hyperlink{art99}{artículo 99º} del presente Estatuto. Sin
  perjuicio de lo anteriormente expuesto, la Junta Representativa
  oficiará como órgano consultivo cada vez que la Comisión Directiva lo
  requiera, teniendo en este caso sus resoluciones un carácter no
  vinculante.
\end{itemize}

\begin{itemize}
\item
  \textbf{Art. 105º} La Junta Representativa se compondrá de seis (6)
  socios titulares y tres (3) suplentes, mayores de veinticinco y cinco
  (25) años de edad que deberán ser socios vitalicios, vitalicios
  protectores o plenos, con una antigüedad inmediata y consecutiva de
  cinco (5) años y que se hallaren al corriente en sus obligaciones con
  Tesorería. Resultarán elegidos en la misma oportunidad que la Comisión
  Directiva y la Comisión Revisora de Cuentas. Para el caso de los
  miembros titulares, la lista ganadora se adjudicará dos tercios (2/3)
  de sus Miembros, en tanto que la primera minoría ocupara el tercio
  (1/3) restante, disputando, en caso de que hubiese más de dos listas,
  este tercio mediante el sistema D'Hondt. Los candidatos a este
  organismo, estarán numerados del primero (1º) al sexto (6º) lugar,
  para el caso de los titulares; y del primero (1º) al tercer (3º) lugar
  para el caso de los suplentes, estableciéndose su rango y/o jerarquía
  dentro del organismo, a los fines de suplantar las vacantes de Vocales
  de Comisión Directiva que establece el
  \protect\hyperlink{art93}{artículo 93º}.
\item
  \textbf{Art. 106º} Los miembros de la Junta Representativa durarán
  tres (3) ejercicios en su mandato y podrán ser reelegidos.
\item
  \textbf{Art. 107º} En caso de acefalía total o imposibilidad de reunir
  a la Junta Representativa, la Comisión Directiva convocará a una
  Asamblea General Extraordinaria dentro de los sesenta (60) días para
  su nueva integración.
\item
  \textbf{Art. 108º} La Junta Representativa funcionará a pedido de
  algún asociado mayor de dieciocho (18) años de edad o de sus tutores o
  representantes legales, en el caso de menores de edad, o por
  convocatoria de la Comisión Directiva, o cuando la Junta, de por sí,
  lo solicitare.
\item
  \textbf{Art. 109º} La solicitud de constitución de la Junta
  Representativa deberá dirigirse a la Comisión Directiva por escrito y
  ésta tiene el deber de proveer de conformidad, dentro de los diez (10)
  días corridos posteriores a la presentación de la solicitud.
\item
  \textbf{Art. 110º} Cada vez que funcione la Junta Representativa
  designará entre sus miembros actuantes, un secretario, quien deberá
  formar expediente y registrará el fallo que será firmado por todos en
  un libro especial. Este organismo será presidido por el Vicepresidente
  2º de la Comisión Directiva, y en caso de su ausencia u otro
  impedimento, este organismo designará por simple mayoría el miembro
  que la presidirá.
\item
  \textbf{Art. 111º} La Junta Representativa puede despachar un asunto
  con la simple mayoría de los votos de sus miembros, dando fuerza a sus
  resoluciones, pero en todos los casos, sus integrantes deberán dejar
  constancia escrita de los fundamentos del voto que emitiesen. Si por
  circunstancias especiales un miembro se excusa de intervenir, su lugar
  será cubierto por uno de los suplentes de acuerdo al orden de
  designación.
\item
  \textbf{Art. 112º} La Junta Representativa resolverá los asuntos
  exclusivamente de la incumbencia para la que fue constituida a pedido
  de un asociado, o de la Comisión Directiva, poseyendo para este caso
  los atributos necesarios para realizar su cometido.
\item
  \textbf{Art. 113º} Las resoluciones de la Junta Representativa serán
  apelables ante la primera Asamblea Ordinaria, pero sus fallos
  subsistirán mientras tanto.
\item
  \textbf{Art. 114º} La Junta Representativa asumirá, en caso de
  acefalía total o definitiva, la administración provisional de la
  Institución, siguiendo el procedimiento establecido en el
  \protect\hyperlink{art67}{art. 67º}.
\item
  \textbf{Art. 115º} Son prerrogativas especiales de la Junta
  Representativa, designar los miembros que compondrán la Junta
  Electoral; así como entender en los casos de cancelación de
  funcionamiento de filiales, a pedido de la Comisión Directiva, como lo
  establece el \protect\hyperlink{art127}{artículo 127º}.
\end{itemize}

\chapter{DE LA COMISIÓN REVISORA DE
CUENTAS}\label{de-la-comision-revisora-de-cuentas}

\begin{itemize}
\item
  \textbf{Art. 116º} La fiscalización social estará a cargo de una
  Comisión Revisora de Cuentas ad-honorem, cuyos miembros serán elegidos
  e n el mismo acto fijado para la elección de la Comisión Directiva y
  Junta Representativa. Estará compuesta por tres (3) miembros titulares
  y dos (2) miembros suplentes, que durarán tres (3) ejercicios en sus
  funciones, podrán ser reelectos y se elegirán de la siguiente manera:

  \begin{itemize}
  \item
    \begin{enumerate}
    \def\labelenumi{\alph{enumi})}
    \tightlist
    \item
      Los primeros dos (2) miembros titulares y el primer miembro
      suplente corresponderán respectivamente a los tres miembros
      titulares propuestos por la lista de candidatos que resultare
      triunfante en las elecciones.
    \end{enumerate}
  \item
    \begin{enumerate}
    \def\labelenumi{\alph{enumi})}
    \setcounter{enumi}{1}
    \tightlist
    \item
      El tercer miembro titular y el segundo miembro suplente,
      corresponderán respectivamente a la lista de candidatos que
      resulte segunda en las elecciones, siempre que esta lista obtenga
      más del quince por ciento (15\%) de los votos válidos.
    \end{enumerate}
  \end{itemize}
\item
  \textbf{Art. 117º} Los miembros de la Comisión Revisora de Cuentas
  deberán tener más de cuatro (4) años de antigüedad inmediata y no
  interrumpida, en la categoría o en la suma de ellas como socio
  vitalicio o pleno, hallándose al corriente en sus obligaciones con
  Tesorería. Las decisiones de la Comisión Revisora de Cuentas, se
  tomarán por simple mayoría.
\item
  \textbf{Art. 118º} Son atribuciones y deberes de la Comisión Revisora
  de Cuentas:

  \begin{itemize}
  \item
    \begin{enumerate}
    \def\labelenumi{\alph{enumi})}
    \tightlist
    \item
      Vigilar y exigir el cumplimiento de las disposiciones
      estatutarias.
    \end{enumerate}
  \item
    \begin{enumerate}
    \def\labelenumi{\alph{enumi})}
    \setcounter{enumi}{1}
    \tightlist
    \item
      Fiscalizar cuentas, balances, informes y memorias, verificar
      fondos, valores y documentos de la Entidad.
    \end{enumerate}
  \item
    \begin{enumerate}
    \def\labelenumi{\alph{enumi})}
    \setcounter{enumi}{2}
    \tightlist
    \item
      Examinar actas y solicitar copias de ellas.
    \end{enumerate}
  \item
    \begin{enumerate}
    \def\labelenumi{\alph{enumi})}
    \setcounter{enumi}{3}
    \tightlist
    \item
      Ilustrar a la Asamblea de cómo ha cumplido su misión si lo
      estimare necesario, e informar sobre cualquier incumplimiento
      estatutario si lo hubiere.
    \end{enumerate}
  \item
    \begin{enumerate}
    \def\labelenumi{\alph{enumi})}
    \setcounter{enumi}{4}
    \tightlist
    \item
      Inspeccionar los gastos, confrontar ingresos y egresos con las
      autorizaciones o constancias que lo justifiquen y documenten.
    \end{enumerate}
  \item
    \begin{enumerate}
    \def\labelenumi{\alph{enumi})}
    \setcounter{enumi}{5}
    \tightlist
    \item
      Examinar la contabilidad de la Institución.
    \end{enumerate}
  \item
    \begin{enumerate}
    \def\labelenumi{\alph{enumi})}
    \setcounter{enumi}{6}
    \tightlist
    \item
      Visar los balances anuales de la Tesorería.
    \end{enumerate}
  \item
    \begin{enumerate}
    \def\labelenumi{\alph{enumi})}
    \setcounter{enumi}{7}
    \tightlist
    \item
      Proponer por escrito a la Comisión Directiva las reformas que crea
      conveniente introducir en el sistema de contabilidad y control.
    \end{enumerate}
  \item
    \begin{enumerate}
    \def\labelenumi{\roman{enumi})}
    \tightlist
    \item
      Asistir con fines consultivos a las reuniones de la Comisión
      Directiva cuando ésta o ellos los estimaren necesarios y en los
      asuntos que sean de su incumbencia, teniendo a tal fin voz pero no
      voto.
    \end{enumerate}
  \item
    \begin{enumerate}
    \def\labelenumi{\alph{enumi})}
    \setcounter{enumi}{9}
    \tightlist
    \item
      Llevar a cabo la confección de un libro de actas, en el cual
      deberán constar los asuntos y las tareas que la Comisión Revisora
      de Cuentas está llevando a cabo, y que podrá ser solicitado por la
      Comisión Directiva trimestralmente o cuando esta lo solicitare.
    \end{enumerate}
  \item
    \begin{enumerate}
    \def\labelenumi{\alph{enumi})}
    \setcounter{enumi}{10}
    \tightlist
    \item
      Llevar un registro de visado que, de manera trimestral, Tesorería
      dará a la Comisión Revisora de Cuentas a los fines de registrar el
      cumplimiento de sus respectivas obligaciones.
    \end{enumerate}
  \end{itemize}
\item
  \textbf{Art. 119º} Las vacantes de la Comisión Revisora de Cuentas se
  cubrirán incorporando los suplentes en el orden de lista. Si no
  obstante la incorporación de los suplentes, la Comisión Revisora de
  Cuentas quedase en minoría o acéfala, se integrará está con los
  miembros de la Junta Representativa, sorteados por la Comisión
  Directiva en presencia de aquellos.
\end{itemize}

\chapter{DE LAS PEÑAS}\label{de-las-penas}

\begin{itemize}
\item
  \textbf{Art. 120º} Se reconocerá oficialmente como PEÑA de nuestra
  Institución cuando como mínimo diez (10) socios deseen organizarse
  como tal en su localidad, pueblo, ciudad o región y eleven la
  solicitud correspondiente a la Comisión Directiva para que resuelva su
  aprobación.
\item
  \textbf{Art. 121º} Serán objetivos a cumplir por las PEÑAS, los
  siguientes:

  \begin{itemize}
  \item
    \begin{enumerate}
    \def\labelenumi{\alph{enumi})}
    \tightlist
    \item
      Difundir el deporte en todas sus manifestaciones.
    \end{enumerate}
  \item
    \begin{enumerate}
    \def\labelenumi{\alph{enumi})}
    \setcounter{enumi}{1}
    \tightlist
    \item
      Recibir, apoyar y alentar a las delegaciones del club que visiten
      la localidad.
    \end{enumerate}
  \item
    \begin{enumerate}
    \def\labelenumi{\alph{enumi})}
    \setcounter{enumi}{2}
    \tightlist
    \item
      Apoyar y colaborar con la Comisión Directiva en toda iniciativa
      que esta promueva.
    \end{enumerate}
  \item
    \begin{enumerate}
    \def\labelenumi{\alph{enumi})}
    \setcounter{enumi}{3}
    \tightlist
    \item
      Fomentar la incorporación de socios plenos en toda su zona de
      actuación.
    \end{enumerate}
  \item
    \begin{enumerate}
    \def\labelenumi{\alph{enumi})}
    \setcounter{enumi}{4}
    \tightlist
    \item
      Estimular todo emprendimiento que redunde en el engrandecimiento
      de nuestra Institución.
    \end{enumerate}
  \end{itemize}
\item
  \textbf{Art. 122º} La representación y funcionamiento de las Peñas que
  se constituyan deberán estar compuestas por personas con domicilio
  fijado en la localidad que obligatoriamente deberán ser socios de
  nuestra Institución. Dicha comisión estará integrada por: un Socio
  Referente 1º, un Socio Referente 2º y un Socio Referente 3º, cuyas
  funciones serán ejercidas ad-honorem.
\end{itemize}

\begin{itemize}
\item
  \textbf{Art. 123º} El valor de la cuota de los socios de las Peñas,
  será inferior a la cuota de socio pleno en un porcentaje que la
  Comisión Directiva fijará en correspondencia con lo dispuesto en el
  \protect\hyperlink{art68}{artículo 68º} inc. m) del presente Estatuto.
\item
  \textbf{Art. 124º} La cuota de los socios a distancia podrán abonarse
  de acuerdo a como lo disponga la Representación de la peña quien
  deberá rendir lo recaudado a Tesorería General del Club, antes del día
  15 de cada mes, en forma personal, transferencia bancaria o cheque.
\item
  \textbf{Art. 125º} Las Peñas podrán recibir donaciones, organizar
  rifas, cenas, eventos deportivos, con el fin de recaudar fondos para
  su mantenimiento y proyección. En todos los casos solicitarán
  autorización a la Comisión Directiva y rendirán un informe escrito y
  firmado por sus representantes acerca de los Ingresos y Gastos en
  forma trimestral.
\item
  \textbf{Art. 126º} Las Peñas deberán redactar su propio reglamento de
  funcionamiento interno que será elaborado dentro del marco que fij a
  el estatuto del club y elevado a la Comisión Directiva para su
  aprobación.
\end{itemize}

\begin{itemize}
\tightlist
\item
  \textbf{Art. 127º} A propuesta de la Comisión Directiva, la Junta
  Representativa podrá cancelar el funcionamiento de la Peña constituida
  cuando así lo estime conveniente, para lo cual deberá contar con el
  voto positivo de los dos tercios (2/3) de sus integrantes.
\end{itemize}

\hypertarget{cap24}{\chapter{DE LAS ELECCIONES}\label{cap24}}

\section{De la Junta Electoral}\label{de-la-junta-electoral}

\begin{itemize}
\tightlist
\item
  \textbf{Art. 128º} La preparación, dirección y contralor de los
  comicios estará a cargo de una Junta Electoral, integrada de la
  siguiente forma:

  \begin{itemize}
  \item
    \begin{enumerate}
    \def\labelenumi{\alph{enumi})}
    \tightlist
    \item
      Un presidente.
    \end{enumerate}
  \item
    \begin{enumerate}
    \def\labelenumi{\alph{enumi})}
    \setcounter{enumi}{1}
    \tightlist
    \item
      Un Secretario.
    \end{enumerate}
  \item
    \begin{enumerate}
    \def\labelenumi{\alph{enumi})}
    \setcounter{enumi}{2}
    \tightlist
    \item
      Un Secretario de Actas.
    \end{enumerate}
  \end{itemize}
\end{itemize}

Cada uno de estos tres miembros integrantes se elegirá nominalmente en
reunión de la Junta Representativa, citada a tal efecto en forma
fehaciente con no menos de 48 hs de anticipación. El número de votos
requeridos para ser elegido miembro de la Junta Electoral, será del 70\%
de los miembros de la Junta Representativa presentes en la reunión.
Además integrarán la Junta Electoral, un miembro por cada una de las
listas participantes de la elección y que será ratificado o no cuando su
lista complete los requisitos establecidos en el
\protect\hyperlink{art135}{artículo 135º} para su oficialización. La
Junta Electoral será designada cuarenta y cinco (45) días corridos
anteriores a la realización de los comicios.

\begin{itemize}
\item
  \textbf{Art. 129º} El cargo de miembro de la Junta Electoral es
  inexcusable, debiendo estar empadronado.
\item
  \textbf{Art. 130º} La Junta Electoral se dictará su reglamento inter
  no, a menos que, con anterioridad, no hubiese sido aprobado uno por
  Asamblea de Socios y entenderá todo lo relativo a la preparación de
  listas, acto electoral, escrutinio definitivo, proclamación de los
  electos, a cuyo efecto tomará cuantas medidas sean necesarias, siempre
  que no contraríen el presente Estatuto, debiendo la Comisión Directiva
  poner a su disposición el personal que estime oportuno.
\end{itemize}

\section{Del Padrón de socios}\label{del-padron-de-socios}

\begin{itemize}
\item
  \textbf{Art. 131º} La Comisión Directiva, con cuarenta y cinco (45)
  días corridos de antelación a todo comicio ordinario, confeccionará el
  padrón de socios con derecho a voto. Gozarán de esta prerrogativa los
  socios vitalicios y los socios plenos que cuenten con una antigüedad
  no inferior a un (1) año, contando retrospectivamente, desde la fecha
  fijada para las elecciones. Para el cómputo de dicha antigüedad se
  tendrá en cuenta --indistintamente- el tiempo transcurrido como socio
  pleno, cadete o integrante de grupo familiar.
\item
  \textbf{Art. 132º} El padrón electoral reunirá los siguientes
  requisitos:

  \begin{itemize}
  \item
    \begin{enumerate}
    \def\labelenumi{\alph{enumi})}
    \tightlist
    \item
      Se hará por orden alfabético, indicando número de socio y ultimo
      domicilio registrado, llevando además una columna para constancia
      de la emisión del voto y otra para observaciones.
    \end{enumerate}
  \item
    \begin{enumerate}
    \def\labelenumi{\alph{enumi})}
    \setcounter{enumi}{1}
    \tightlist
    \item
      Se imprimirán las copias necesarias para ser expuestas en la Sede
      Social, para ser entregadas a la Junta Electoral, a las
      Agrupaciones de socios que intervengan en la elección, y para las
      autoridades del comicio en cada mesa.
    \end{enumerate}
  \item
    \begin{enumerate}
    \def\labelenumi{\alph{enumi})}
    \setcounter{enumi}{2}
    \tightlist
    \item
      Las modificaciones que correspondiera introducir al padrón una vez
      impreso, responderán --única y exclusivamente- a omisiones o
      rectificaciones debidamente comprobadas hasta tres (3) días antes
      de la elección, debiendo en todos los casos labrarse un acta ante
      la Junta Electoral y dando intervención a los apoderados de las
      listas participantes en los comicios.
    \end{enumerate}
  \end{itemize}
\item
  \textbf{Art. 133º} Están excluidos del padrón electoral los socios que
  en el momento de confección se encuentren en uso de licencia o
  cumpliendo una pena disciplinaria impuesta por la Comisión Directiva.
  Los inhabilitados para votar que después de la formación del padrón y
  depuración entraren en el ejercicio de este derecho, presentarán una
  nota solicitando su inclusión a la Junta Electoral, quien determinará
  --en cada caso- sobre la procedencia del pedido.
\end{itemize}

\section{De la Oficialización de las
Listas}\label{de-la-oficializacion-de-las-listas}

\begin{itemize}
\tightlist
\item
  \textbf{Art. 134º} Diez (10) días antes, por lo menos, de la fecha de
  realización de la elección, deberán ser presentadas las listas
  completas de candidatos para ocupar los cargos de la Comisión
  Directiva, Junta Representativa y Comisión Revisora de Cuentas, las
  cuales deberán llevar una denominación determinada que no sea igual a
  otra ya presentada. Ocho (8) días antes, por lo menos, a la
  realización de la elección, la Junta Electoral deberá oficializar las
  listas presentadas. En los casos precedentes los plazos se contarán en
  días co rridos y hasta la cero (0) hora del día de la elección.
\end{itemize}

\begin{itemize}
\item
  \textbf{Art. 135º} La oficialización de las listas presentadas no
  podrá ser negada por la Junta Electoral siempre que llenen los
  siguientes requisitos:

  \begin{itemize}
  \item
    \begin{enumerate}
    \def\labelenumi{\alph{enumi})}
    \tightlist
    \item
      Que las solicitudes vengan firmadas por no menos de cien (100)
      socios electores y que no figuren como candidatos debiendo
      contener el nombre de las personas que actuarán como Apoderados
      Generales y fiscales de cada mesa.
    \end{enumerate}
  \item
    \begin{enumerate}
    \def\labelenumi{\alph{enumi})}
    \setcounter{enumi}{1}
    \tightlist
    \item
      Que todos los candidatos hayan aceptado el carácter de tal, debie
      ndo la aceptación firmada acompañarse a la solicitud.
    \end{enumerate}
  \item
    \begin{enumerate}
    \def\labelenumi{\alph{enumi})}
    \setcounter{enumi}{2}
    \tightlist
    \item
      Que los firmantes de la solicitud no hayan suscrito otra
      presentación. Si figurara alguno en estas condiciones, su firma no
      será tenida en cuenta.
    \end{enumerate}
  \end{itemize}
\item
  \textbf{Art. 136º} Todo socio que hubiera aceptado su inclusión como
  candidato y no se hiciere cargo de su puesto, en caso de ser electo,
  quedará inhabilitado para integrar la Comisión Directiva, Junta
  Representativa y la Comisión Revisora de Cuentas por el término de
  cuatro (4) años.
\item
  \textbf{Art. 137º} Los votos que se dieran a socios cuyas candidaturas
  no hubieran sido oficializadas no serán computados. Sin embargo, se
  computará el voto dado a favor de cualquier candidato y cualquiera sea
  la boleta en que esté incluido y haya sido ésta oficializada y siempre
  que sean para el cargo postulado oficialmente.
\item
  \textbf{Art. 138º} Si en una boleta, al practicarse el escrutinio,
  figuran más nombres de los que corresponde a los cargos que se eligen,
  se computarán solamente los primeros inscriptos, hasta completar el
  número establecido.
\item
  \textbf{Art. 139º} En caso de que uno o más candidatos incluidos en
  cualquier lista no reuniera los requisitos estatutarios, la lista será
  devuelta para su ulterior presentación, corregida, dentro del plazo
  improrrogable de veinticuatro (24) horas.
\end{itemize}

\section{Del Acto Electoral}\label{del-acto-electoral}

\begin{itemize}
\item
  Art. 140º** Las elecciones se efectuarán en el local del Club , o en
  el que se designare la Comisión Directiva y en el día fijado, desde
  las ocho (8) horas hasta las dieciocho (18) horas y se constituirán
  tantas mesas receptoras de votos como grupos de doscientos cincuenta
  (250) socios o fracción tenga la Institución, con derecho a voto.
  Estas mesas serán presididas, cada una, por un socio, actuando otro
  como auxiliar, debiendo ser todos ellos electores y designados por la
  Junta Electoral y fiscales titulares por cada lista oficializada. La
  mesa que a la hora fijada no contase con los fiscales titulares,
  funcionará con el Presidente o el suplente del mismo y a falta de
  estos con el socio que designe la Junta Electoral.
\item
  Art. 141º** El orden en el lugar de los comicios, lo mantendrán la
  Junta Electoral con los asociados o empleados que solicitare a la
  Comisión Directiva, del cual es responsable ante la misma.
\item
  \textbf{Art. 142º} Inmediato a cada mesa, en el cuarto oscuro, el día
  de los comicios, deberán hallarse la o las boletas de los distintos
  candidatos, a disposición de cada votante, confeccionadas sobre papel
  blanco en tinta negra, sin propaganda alguna, donde se lea con
  claridad el nombre, apellido y número de asociado de los candidatos,
  como así también el cargo a que se postula.
\item
  \textbf{Art. 143º} La votación será secreta, y cada votante, previo al
  depósito en la urna de la boleta, en sobre cerrado, exhibirá
  indefectiblemente su carnet societario y documento nacional de
  identidad.
\item
  \textbf{Art. 144º} El presidente de la mesa estampará un sello con l a
  leyenda ``votó'' en el padrón. En caso de identidad dudosa, o que el
  votante no muestre su documento nacional de identidad o carnet de
  socio, el Presidente o los fiscales podrán observar o impugnar el
  voto.
\item
  \textbf{Art. 145º} Los socios votarán por Presidente, Vicepresidente
  1º, Vicepresidente 2º, Secretario General, Tesorero y once (11)
  candidatos a Vocales, por orden de lista, todos ellos para integrar la
  Comisión Directiva; seis (6) miembros titulares y tres (3) suplentes
  para integrar la Junta Representativa; y tres (3) miembros titulares y
  dos (2) miembros suplentes, para integrar la Comisión Revisora de
  Cuentas.
\end{itemize}

\section{Del Escrutinio}\label{del-escrutinio}

\begin{itemize}
\item
  \textbf{Art. 146º} A las dieciocho (18) horas, se clausurará la
  entrada a los comicios por el Presidente de la Junta Electoral y una
  vez que hayan votado los socios que aún se hallaren en el local, el
  Presidente de cada mesa procederá a abrir la urna y realizar el
  recuento de votos y de sobres. Una vez realizada esta operación se
  colocará nuevamente todas las boletas y sobres en l a urna, se cerrará
  ésta, se tapará la boca con una faja de papel que al efecto proveerá
  la Junta Electoral y llevará la firma de la autoridad de cada mesa.
  Acto seguido, el Presidente de la mesa labrará un acta en donde
  constará el número de sufragios emitidos, votos anulados, observados,
  como así también todas las observaciones que creyere conveniente
  anotar. Dichas actas, conjuntamente con las urnas, será llevadas por
  el Presidente de la mesa, que podrá ser acompañado por los fiscales de
  cada mesa, al lugar donde se constituya la mesa escrutadora, que
  estará integrada por los miembros de la Junta Electoral. Una vez
  recibidas todas las urnas, la junta escrutadora iniciará el recuento
  de votos y el escrutinio definitivo. Inmediatamente de finalizado el
  mismo, la Junta Electoral proclamará los candidatos triunfantes,
  labrándose el acta respectiva, la que será elevada al Presidente de la
  Comisión Directiva, para los efectos consiguientes.
\item
  \textbf{Art. 147º} En caso de que dos o más listas hayan obtenido el
  mismo número de votos, en calidad de triunfantes, se convocará dentro
  de los quince (15) días siguientes, a una nueva elección con la
  intervención exclusiva de las agrupaciones que hayan igualado el
  número de sufragios.
\end{itemize}

\begin{itemize}
\item
  \textbf{Art. 148º} La Junta Electoral, en el caso de que fuera
  oficializada una sola lista procederá en el acto de apertura del
  comicio, a proclamarla, declarando así no haber necesidad del acto
  eleccionario.
\item
  \textbf{Art. 149º} Dentro de los diez (10) días subsiguientes a la
  proclamación de los candidatos electos, éstos tomarán posición de sus
  cargos, cesando indefectiblemente los miembros salientes.
\end{itemize}

\chapter{DE LA REVOCATORIA DE LOS
MANDATOS}\label{de-la-revocatoria-de-los-mandatos}

\begin{itemize}
\item
  \textbf{Art. 150º} Cuando los socios dispongan aplicar el inciso m)
  del \protect\hyperlink{art32}{artículo 32º}, deberán solicitar por
  escrito la realización de un referéndum que determine la continuidad o
  no de las autoridades de la Institución. La solicitud ante el
  Presidente y por su intermedio a la Junta Representativa (con copia a
  Inspección de Personas Jurídicas), deberá estar acompañada por las
  firmas de un número de socios mayor al número de votos obtenidos por
  la lista mayoritaria en la última elección y no menor al 50\% del
  total de socios que hayan sufragado. Además los firmantes deberán
  mantener la proporción de plenos y vitalicios de la última elección.
  Todas las firmas deberán estar certificadas por escribano público.
\item
  \textbf{Art. 151º} Tomada la decisión de convocar a referéndum, la
  Junta Representativa en un plazo no mayor de cinco días fijará la
  fecha de realización que no podrá ser mayor a los 60 días corridos ni
  menor a los treinta (30) días corridos. Designará una Junta Electoral
  que estará integrada por un presidente, un secretario, un secretario
  de actas, y por dos socios, uno coincidente con la propuesta que se
  lleva a referéndum y otro que no coincida. En este caso, los miembros
  de esta Junta Electoral deberán cumplir el requisito de tener diez
  (10) años de antigüedad ininterrumpida y como mínimo uno (1) de sus
  integrantes preferentemente deberán tener título de abogado.
\item
  \textbf{Art. 152º} Para que el referéndum tenga validez se establece
  que el número de votantes debe ser mayor al número total de votantes
  de la última elección de autoridades del club.
\item
  \textbf{Art. 153º} Se revocará el mandato de las autoridades cuando el
  número de votos por la revocatoria, sea mayor al 50 por ciento y a su
  vez no menor al número de votos que haya obtenido la lista mayoritaria
  en la última elección.
\item
  \textbf{Art. 154º} Si el resultado fuere la revocatoria del mandato de
  las autoridades, la Junta Representativa deberá convocar en un plazo
  no mayor a cinco (5) días corridos, a elecciones generales en un plazo
  no mayor de sesenta (60) días corridos y se regirá por todas las
  disposiciones establecidas en el capítulo de las elecciones.
\end{itemize}

\chapter{DISPOSICIONES
COMPLEMENTARIAS}\label{disposiciones-complementarias}

\begin{itemize}
\item
  \textbf{Art. 155º} Estos estatutos entrarán a regir inmediatamente d e
  ser aprobados por la Asamblea convocada al efecto y por el Superior
  Gobierno de la Provincia, quedando la Comisión Directiva autorizada
  para solicitar su aprobación y aceptar modificaciones que este último
  hiciera.
\item
  \textbf{Art. 156º} A partir de la aprobación de este Estatuto, se
  establece el ``Día del Club'', el que será festejado anualmente en la
  fecha más próxima a la del Aniversario de la Institución (8 de
  agosto). En esa fecha, todos los asociados estarán obligados, para
  gozar de los espectáculos organizados por el Club, cualquiera fuere su
  naturaleza y demás beneficios que otorga la Institución, a abonar una
  contribución no menor de cincuenta (50) por ciento de la entrada
  general a los mencionados espectáculos.
\item
  \textbf{Art. 157º} Queda la Comisión Directiva autorizada a fijar
  precios a los espectáculos deportivos y sociales que realizare en las
  instalaciones del Club, pudiendo establecer una contribución
  obligatoria a los asociados, no mayor del cincuenta (50) por ciento de
  la entrada a dichos espectáculos.
\item
  \textbf{Art. 158º} Sin consentimiento previo de la Comisión Directiva,
  no podrá hacerse en el Club, suscripciones o cuestionarios de ninguna
  índole.
\item
  \textbf{Art. 159º} Es potestad de la Comisión Directiva decidir el
  llamado a elecciones de manera anticipada, procediéndose a la
  realización del proceso eleccionario tal como está establecido en el
  articulado del \protect\hyperlink{cap24}{Capítulo XXIV} del presente
  Estatuto.
\item
  \textbf{Art. 160º} El personal administrativo, profesional o que
  perciba remuneración por cualquier concepto, sea o no socio, deberá
  mantener absoluta imparcialidad y prescindencia frente a los
  movimientos electorales de los asociados, bajo pena de cesantía.
  Además los que sean socios no podrán votar en Asambleas, Referéndum o
  elecciones ni ser electos para ningún cargo estatutario, hasta dos
  años después de haber cesado su cargo rentado.
\item
  \textbf{Art. 161º} Es incompatible el cargo de Presidente y Secretario
  General o Presidente y Tesorero General, cuando entre los titulares de
  dichos cargos existiese una relación de parentesco hasta el cuarto
  grado de consanguinidad y segundo de afinidad.
\item
  \textbf{Art. 162º} El presente estatuto deroga y anula toda
  disposición anterior que a él se oponga.
\end{itemize}


\end{document}
